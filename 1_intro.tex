\section{はじめに}
ベントナイト緩衝材の変形や物質輸送特性を正確に評価するためには,その
主成分であるモンモリロナイトの膨潤について十分理解することが必要となる.
モンモリロナイトは水分に接触すると,粘土層間の陽イオンに水和する
形で吸水する.このとき,積層した粘土層の層間距離は拡がり,粘土含水系全体
としても膨潤を起こす.膨潤にともない粘土の積層状況や間隙構造は変化する.
そのため,膨潤を伴う水分の浸透は,結果として透水性を変化させる可能性がある.
また粘土含水系に応力を与えて排水させる,すなわち圧排水するためには, 
膨潤圧に抗して粘土を圧縮する必要がある.このように,膨潤は水分浸透や
圧密変形と密接な関係がある.さらに,間隙水分布や間隙ネットワークの構造は,
熱や物質の輸送にも影響するため,膨潤挙動はベントナイトの物性や機能を評価する
上で重要な意味を持つ.

ベントナイトの膨潤や水分浸透の挙動を粘土鉱物スケールで直接的に観測することは難しい.
これは,粘土鉱物や粘土中の間隙が非常に小さく複雑な形状をしているためである.
従って,粘土鉱物スケールでの水和や膨潤の挙動の詳細知るためには,
分子動力学(MD)法を始めとする計算科学的なアプローチをとる必要がある.
ただし,物質を構成する原子一つ一つに自由度を与えて運動方程式を解く全原子MD法は
計算負荷が高く,多数の粘土鉱物と水分子で構成された粘土含水系を扱うことは難しい.
このことを踏まえて本共同研究では,複数の原子や分子を一つの粒子(粗視化粒子)として扱う
粗視化分子動力学法(coarse-grained molecular dynamics method: CGMD法)の開発に
取り組んできた.CGMD法では,モンモリロナイト粘土分子の単位構造と,そこに水和した
間隙水を一つの粗視化粒子で表現してモデルの自由度を削減することで,より多数の
粘土分子の系で凝集や変形の挙動を調べることができる.
昨年度までの研究では,CGMDシミュレーションにより粘土含水系の組織構造モデルを作成して
X線回折パターンを合成し,実験でも観測される観測方向で回折ピークが得られる
こと等を示してきた.また,シミュレーション結果から動径分布関数を評価すれば,
粘土分子の積層挙動や積層数の見積りが得られるなど,粘土含水系組織の形成機構を
理解する上で有用な知見が得られることを示してきた.
しかしながら,これまでのシミュレーションでは間隙水分量は一定としているため,
膨潤や膨潤圧の発生,系が平衡状態にあるときに保持される水分量をシミュレーション
によって求めることはできなかった.
そこで本年度の研究では,粘土含水系に水分の出入りがある場合にも対応できるよう,
CGMD法の拡張を行う.これにより,系外から水分を浸透させることで膨潤や
膨潤圧を発生させることができるようになる.これは,指定された温度や体積,
化学ポテンシャルの下で粘土含水系に保持される水分量を,
CGMD法によって決定することができるようになることを意味する.

本稿の以下では,はじめにCGMD法のモデルとアルゴリズムについて述べる.
その際,水分の移動と水分量の変化に関わるエネルギーの取扱について詳細を示す.
次に,水分量変化を許容した新しいCGMD法によるシミュレーションの結果を示す.
具体的には,乾燥密度が1.6g/cm$^3$程度のモデルを水分量一定の条件でCGMD法に
よって作成し,それを初期モデルとして水分の出入りを許す条件で緩和シミュレーション
を行う.緩和は温度と体積,化学ポテンシャルが一定の条件で,
いくつかの化学ポテンシャルについて行う.
その結果,同一の初期モデルから緩和シミュレーションを開始しても,化学
ポテンシャルの大小によって吸水,排水のいずれも生じうることを示す.
以上の結果を示した後,本年度の研究に関するまとめと今後の課題について述べる.
