\section{水分量変化を許容したCGMDシミュレーション}
最後に,定温定積かつ化学ポテンシャル一定の元で行った,CGMDシミュレーションの結果を示す.
ユニットセルのサイズは初期モデルの状態で固定し,温度を300[K]に保ったまま
一定の化学ポテンシャル$\tilde \mu$のもとで1[ns]の間,水分の出入りを
許す状態で粘土含水系を緩和させる.

図\ref{fig:fig4}は,$\tilde \mu=0$の場合について組織構造の変化を示したものである.
この場合,水分量の増減によるエネルギーの変化はなく,水和エネルギーの減少が
ポテンシャルエネルギーの増加を上回る限り系内部の水分量は増加する.
その結果、積層した粘土層間の距離が大きくなり,かつ層間距離が均一化する様子が見られる.
例えば,図\ref{fig4}-(a)において(i)と(ii)の破線で囲んだ箇所は,当初,層間距離が
にばらつきが見られるが,これらの箇所吸水により最終的には層間距離がほぼ一定の値に
揃っている.また,粘土分子が折りたたまれて出来た空隙部分も水分の増加に伴い,その
大きさは次第に小さくなっている.
%
これに対して図\ref{fig:fig5}は,化学ポテンシャルを
\begin{equation}
	\frac{\tilde \mu}{u'(0)}=
	\frac{\tilde \mu}{\gamma u_\infty}=\frac{1}{5}
	\label{eqn:mu02}
\end{equation}
とした場合の結果を示したものである.式(\ref{eqn:mu02})は,
水和エネルギー$u(s)$の勾配の最大値$u'(0)=\gamma u_{\infty}$を基準にして
化学ポテンシャルを与えることを意図したものである.
ここでは,水分量の変動によるエネルギーの増減が,
水和エネルギーの増減を凌駕して完全な排水が起こることが無い程度の値に
化学ポテンシャルの大きさを設定している.
実際,図\ref{fig:fig5}の結果では,一部の水分が排出されることで
空隙が大きくなる方向へ組織構造が変化しているが,水分量の変化は
あまり大きくないため,粘土分子の相対的な配置や屈曲の変化は劇的なものではない.
%図$\tilde \mu=0$の場合よりも顕著な組織構造の変化がおきているが,
%水分量の変化粘土分子の全体的な分布には大きな変化が置きていない.
ただし,詳しくみれば,水分量の減少によって生まれたスペースを使って粘土分子が
若干移動することで,屈曲のが解消される箇所や空洞が統合あるいは消失する様子が
示されている.
%--------------------
\begin{figure}[h]
	\begin{center}
	\includegraphics[width=1.0\linewidth]{Figs/fig4.eps} 
	\end{center}
	\caption{
		定積,定温,化学ポテンシャル一定での緩和挙動($\tilde \mu =0$の場合).
	} 
	\label{fig:fig4}
\end{figure}
%--------------------
%--------------------
\begin{figure}[h]
	\begin{center}
	\includegraphics[width=1.0\linewidth]{Figs/fig5.eps} 
	\end{center}
	\caption{
		定積,定温,化学ポテンシャル一定での緩和挙動($\tilde \mu =\frac{1}{5}\gamma u_{\infty}$の場合).
	} 
	\label{fig:fig5}
\end{figure}
%--------------------
次に,各粗視化粒子が有する水分量を図\ref{fig:fig6}に示す.
この図は,横軸に粗視化粒子番号を,縦軸に水和水層の厚さ$s_i^+$と$s^i_-$を示した
ものである.一つの分子を構成する粗視化粒子には連続する粒子番号が振られており,
この図では,各粒子が属する分子が変化するたびに異なる色の実線で水和水層の厚さを
プロットしている.なお$s^+_i$は実線で,$s^-_i$は破線で表示しているが,
両者はほとんどの場合近い値となっておりグラフ上ではほとんど区別がつかなくなっている.
(a),(b),(c)はそれぞれ,化学ポテンシャル一定での緩和の開始時,$\tilde \mu =0$と
$\tilde \mu =\frac{1}{5}\gamma u_{\infty}$での緩和を終了させた時点での結果を示している.
図\ref{fig:fig7}はこれらの結果を$s^\pm_i$のヒストグラムとして表示したものである。
これらの図から明らかな通り,当初水分量が大きく偏った状態にあるものの,
緩和後には水分が粒子に概ね均等に配分されている.特に,化学ポテンシャルの値が相対的
に大きな図\ref{fig:fig6},\ref{fig:fig7}-(c)のケースでは,全ての粒子で
一定値$s^{^pm}_i\simeq =0.235$[nm]に近づいており、これは1.5層膨潤程度に相当する.
なお,$\tilde \mu=0$の場合には,いくつかの粘土分子で平均値よりも明らかに大きな
$s^\pm_i$の値を取るものがある.これは,緩和開始時の水分分布の偏りが残留したもので,
粘土分子影響を受けたもので,水分の増加にともない,粘土分子が移動する余地がほとんど無く
層間距離をほとんど変化させることができない箇所に水分が充填された結果と考えることができる.
%--------------------
\begin{figure}[h]
	\begin{center}
	\includegraphics[width=0.8\linewidth]{Figs/fig6.eps} 
	\end{center}
	\caption{
		各粗視化粒子における水和水層の厚さ$s_i^+,s_i^-$.
		(a)緩和開始時,(b)$\tilde \mu=0$, (c)$\tilde \mu =\frac{1}{5}\gamma u_\infty.$
	} 
	\label{fig:fig6}
\end{figure}
%--------------------
%--------------------
\begin{figure}[h]
	\begin{center}
	\includegraphics[width=0.6\linewidth]{Figs/fig7.eps} 
	\end{center}
	\caption{
		各粗視化粒子における水和水層の厚さ$s_i^+,s_i^-$のヒストグラム.
		(a)緩和開始時,(b)$\tilde \mu=0$, (c)$\tilde \mu =\frac{1}{5}\gamma u_\infty.$
	} 
	\label{fig:fig7}
\end{figure}
%--------------------
ここでは体積が一定の条件で緩和を行っている。
そのため、緩和の過程で吸水が起きる場合には膨潤圧が,排水が起きる場合は
減圧が生じる.
図\ref{fig:fig9}と図\ref{fig:fig10}はこのことを示したものである。
これらの図には、温度と応力,エネルギーの推移が示されている。
温度は300[K]に保たれているが,水分の移動により粗視化粒子の運動状況が変化する
ため、300[K]前後で10度程度のゆらぎが見られる。
応力は,せん断力成分$\sigma_{12}$は化学ポテンシャルの値によらず
一貫してほぼ零だが,直応力成分$\sigma_{11}$と$\sigma_{22}$は$\tilde \mu=0$では
単調に増加$\tilde \mu =1/2\gamma u_\infty$では減少した後、若干のオーバーシュート
を経て一定値に至る様子が現れている.
この間、エネルギーは,ポテン車ry
%--------------------
\begin{figure}[h]
	\begin{center}
	\includegraphics[width=1.0\linewidth]{Figs/fig9.eps} 
	\end{center}
	\caption{
		化学ポテンシャル一定の元での緩和に伴う
		(a)温度, (b)応力と(c)エネルギーの変化 ($\tilde \mu=0$).
	} 
	\label{fig:fig9}
\end{figure}
%--------------------
%--------------------
\begin{figure}[h]
	\begin{center}
	\includegraphics[width=1.0\linewidth]{Figs/fig10.eps} 
	\end{center}
	\caption{
		化学ポテンシャル一定の元での緩和に伴う(a)温度, (b)応力と (c) エネルギーの変化
		($\tilde \mu =\frac{1}{5}\gamma u_\infty).$
	} 
	\label{fig:fig10}
\end{figure}
%--------------------
%--------------------
\begin{figure}[h]
	\begin{center}
	\includegraphics[width=0.8\linewidth]{Figs/fig11.eps} 
	\end{center}
	\caption{
		caption.
	} 
	\label{fig:fig11}
\end{figure}
%--------------------
%--------------------
\begin{figure}[h]
	\begin{center}
	\includegraphics[width=1.0\linewidth]{Figs/fig12.eps} 
	\end{center}
	\caption{
		caption.
	} 
	\label{fig:fig12}
\end{figure}
%--------------------

