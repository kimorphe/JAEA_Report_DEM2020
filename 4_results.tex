\section{水分量変化を許容したCGMDシミュレーション}
本節では定温定積かつ指定された化学ポテンシャルの下で行った粘土含水系の緩和シミュレーション結果を示す.
緩和シミュレーションの初期モデルには,前節で示した圧縮凝集と冷却過程を経て得られた組織構造モデルを用いる.
ここでは,ユニットセルのサイズを初期モデルの状態で固定,温度を300[K]に設定し,
予め指定した化学ポテンシャル$\tilde \mu$のもとで1[ns]間,CGMDシミュレーションを行う.
\subsection{組織構造の変化}
図\ref{fig:fig4}は,$\tilde \mu=0$として行った緩和シミュレーションで得られた
組織構造を示したものである.この場合,$\tilde \mu=0$であることから,
水分量の増減によるエネルギーの変化$\mu \Delta N$が無い.そのため,
水和エネルギー$U_{hyd}$の減少がポテンシャルエネルギー$U_{LJ}$の増加を上回る限り,
系内の水分量は増加することができる.
図\ref{fig:fig4}に示した結果では,積層した粘土層間の距離が緩和の結果初期状態より
大きくなり,同時に層間距離が均一化する様子が見られる.
例えば,図\ref{fig:fig4}-(a)において(i)と(ii)の破線で囲んだ箇所では,当初,層間距離が
にばらつきが見られる.これらの箇所は吸水により最終的に層間距離がほぼ一様になっている.
また,粘土分子が折りたたまれて出来た空隙部分も水分の増加に伴い,その大きさは次第に
小さくなっている.図\ref{fig:fig4}-(a)に示した矢印はそのような空隙の一例を示したものである.
%
これに対して図\ref{fig:fig5}は,化学ポテンシャルを
\begin{equation}
	\frac{\tilde \mu}{\mu _0}= \frac{1}{5}
	, \ \ 
	\left( 
		\mu_0:=u'(0)=\gamma u_\infty
	\right)
	\label{eqn:mu02}
\end{equation}
とした場合の結果を示したものである.式(\ref{eqn:mu02})は,水和エネルギー勾配$u'(s)$の最大値
$u'(0)=\gamma u_{\infty}$を基準$\mu_0$として化学ポテンシャルを与えることを意図したものである.
これを$\mu=\frac{1}{5}$とした理由は,水分量の変動によるエネルギーの増減$\mu \Delta N$が,
水和エネルギーの変動$\Delta U_{hyd}$を凌駕することなく,適度な排水を生じさせることを意図
したためである.
実際,図\ref{fig:fig5}の結果では,一部の水分が排出されることで空隙が増加する方向へ組織構造が
変化している.詳しくみれば,水分量の減少によって生まれたスペースを使って粘土分子が若干移動することで,
屈曲が解消される箇所(図\ref{fig:fig5}-(c)の(iii)で示した箇所)や,
空洞が消失や統合する様子が示されている(それぞれ,図\ref{fig:fig5}-(b)の(i)と(ii)の箇所).
ただし,排水量はあまり大きくないために,粘土分子の相対的な配置や分子の屈曲状況の変化は
劇的なものではない.
%--------------------
\begin{figure}[h]
	\begin{center}
	\includegraphics[width=1.0\linewidth]{Figs/fig4.eps} 
	\end{center}
	\caption{
		定積,定温,化学ポテンシャル$\tilde \mu =0$での緩和挙動.
	} 
	\label{fig:fig4}
\end{figure}
%--------------------
%--------------------
\begin{figure}[h]
	\begin{center}
	\includegraphics[width=1.0\linewidth]{Figs/fig5.eps} 
	\end{center}
	\caption{
		定積,定温,化学ポテンシャル$\tilde \mu =\frac{1}{5}\mu_0 $での緩和挙動.
	} 
	\label{fig:fig5}
\end{figure}
%--------------------
\subsection{水分分布の変化}
次に,緩和前後での水分分布状況について調べる.
図\ref{fig:fig6}は,各粗視化粒子が有する水分量を,緩和前(初期状態)と
緩和後($\tilde \mu=0$と$\tilde \mu =\frac{\mu_0}{5}$)の場合について示したも
のである.横軸は粗視化粒子番号$i(=1,\dots N)$を,縦軸は水和水層の厚さ($s_i^+,\, s^i_-$)を表す.
一つの粘土分子を構成する粗視化粒子には,連続する粒子番号が与えられている.
そこでこの図では,粒子が属する分子が異なるものになるたびに,異なる色を用いて
水和水層の厚さをプロットしている.従って,同一色の線で描かれた区間が粘土分子
毎の水分量を表している.なお$s^+_i$は実線で,$s^-_i$は破線で表示しているが,
両者はほとんどの場合,非常に近い値となるためグラフ上ではほとんど区別がつかなくなっている.
図\ref{fig:fig7}は同じ水分量のデータを$s^\pm_i$のヒストグラムとして表示したもので,
分子毎の水分量に関する情報は失われるが,水和水層厚さの分布を見るためのものである.
これら図\ref{fig:fig6}と図\ref{fig:fig7}から明らかな通り,水分量は当初大きく
偏った状態にある.このように偏在した水分が,緩和後は概ね均等に配分さている.
特に,化学ポテンシャルの値が相対的に大きな$\tilde \mu= \frac{1}{5}\mu_0$の場合,
水和水層厚は,$s^{\pm}_i\simeq 0.23$[nm]程度の非常に狭い範囲に集中して
分布していることが図\ref{fig:fig7}-(c)に示されている.
なお,これはおよそ1.5層膨潤程度の状態にあることを意味する.
一方,$\tilde \mu=0$の場合,$s^\pm_i$の平均は0.3[nm]で2層膨潤程度となるが,
分布幅は$\tilde \mu =\frac{1}{5}\mu_0$のケースに比べて広く,
いくつかの粘土分子で平均値よりも明らかに大きな水和水層厚となっている.
これは,吸水に伴い粘土分子の相対位置が制限され,相対的に大きな空隙が
残されたまま水分が充填されるためと考えられる.
また,$s^\pm_i$の下限値は0.26[nm]で,ヒストグラムは非対称な形状を示し,
これは水和エネルギー$u(s)$の分布に起因したもので,
物理的には水分量が少ないとき,非常に強い水和が起きることに対応している.
%--------------------
\begin{figure}[h]
	\begin{center}
	\includegraphics[width=0.8\linewidth]{Figs/fig6.eps} 
	\end{center}
	\caption{
		各粗視化粒子における水和水層の厚さ$s_i^+,s_i^-$.
		(a)緩和開始時と,緩和後(b)$\tilde \mu=0$, (c)$\tilde \mu =\frac{1}{5}\mu_0$
		に対する結果.
	} 
	\label{fig:fig6}
\end{figure}
%--------------------
%--------------------
\begin{figure}[h]
	\begin{center}
	\includegraphics[width=0.6\linewidth]{Figs/fig7.eps} 
	\end{center}
	\caption{
		各粗視化粒子における水和水層の厚さ$s_i^+,s_i^-$のヒストグラム.
		(a)緩和開始時と,緩和後(b)$\tilde \mu=0$, (c)$\tilde \mu =\frac{1}{5}\mu_0$
		に対する結果.
	} 
	\label{fig:fig7}
\end{figure}
%--------------------
\subsection{応力とエネルギーの推移}
ここではユニットセル体積を一定としているため,緩和により吸水が起きる場合には
膨潤圧が発生し,排水が起きる場合には応力が減じる.図\ref{fig:fig9}と図\ref{fig:fig10}は
このことを$\tilde \mu=0$と$\frac{1}{5}\mu_0$の場合についてそれぞれ示したもので,
応力の推移と併せて温度とエネルギーの変化も示されている.
温度は300[K]に設定されているが,必ずゆらぎを伴う.
特に,水分の移動により粗視化粒子の運動状況が変化するため,ゆらぎの幅は
水分分布を固定した場合よりも大きくなると予想されるが,図\ref{fig:fig9},図\ref{fig:fig10}の
結果とも,300[K]の周辺で$\pm$15[K]程度のゆらぎにとどまっていることが分かる.
応力は,せん断力成分$\sigma_{12}$は0付近でほぼ変が無い.
一方,直応力成分$\sigma_{11},\sigma_{22}$は互いに等しく等方的な圧力が生じており,
これが緩和により変動する様子が示されている.
$\tilde \mu=0$のときには,吸水によって直応力は増加した後一定値に至っている.
この間,ポテンシャルエネルギー$U_{LJ}$は増加しているが,水分の増加による
水和エネルギー$\Delta U_{hyd}$の減少分が$U_{LJ}$を大きく上回ることで,
系全体のエネルギーが単調に減少して下限値に近づくことが,図\ref{fig:fig9}-(c)に
示されている.このグラフでは,分子間相互作用のポテンシャル$U_{LJ}$と,
水和エネルギーの初期状態からの変化$\Delta U_{hyd}(t)=U_{hyd}(t)-U_{hyd}(0)$,
水分量の変化によるエネルギーの変化
\begin{equation}
	\mu \Delta N= \mu \left(N(t)-N(0) \right) 
%	= \sum_i^n \int_0^t \tilde \mu \frac{d\sigma_i}{dt} dt
	= \sum_i^n\ \sum_{\beta=+,-}\left\{ s_i^\beta(t)-s_i^\beta(0) \right\}
	\label{eqn:mu_dN}
\end{equation}
と,それらの合計を示している.
ここに,$N(t)$は時刻$t$における水分子の総数を意味する.
$\tilde \mu=0$であるため,図\ref{fig:fig9}-(c)で$\mu \Delta N$に変化はない.
%
一方,$\tilde \mu =\frac{1}{2}\mu_0$のときには,直応力は排水によってやや振動しながら低下して
一定値に漸近する.エネルギーに関しては,水分量の減少によるエネルギ−変化$\mu \Delta N$と
水和ネルギー$\Delta U_{hyd}$とも低下し,ポテンシャルエネルギーはわずかに増加している.
$\mu \Delta N$と$\Delta U_{hyd}$がともに減少することは,
水和水層の厚さが大きな箇所からの排水量が,層厚の小さな箇所への給水量を上回ることを
意味している.
% 応力の変化=(運動エネルギー−分子間力)の変化
%--------------------
\begin{figure}[h]
	\begin{center}
	\includegraphics[width=1.0\linewidth]{Figs/fig9.eps} 
	\end{center}
	\caption{
		化学ポテンシャル$\tilde \mu=0$での緩和に伴う(a)温度, (b)応力と(c)エネルギーの変化 .
	} 
	\label{fig:fig9}
\end{figure}
%--------------------
%--------------------
\begin{figure}[h]
	\begin{center}
	\includegraphics[width=1.0\linewidth]{Figs/fig10.eps} 
	\end{center}
	\caption{
		化学ポテンシャル$\tilde \mu =\frac{1}{5}\mu_0$
		での緩和に伴う(a)温度, (b)応力と (c) エネルギーの変化.
	} 
	\label{fig:fig10}
\end{figure}
%--------------------
\subsection{より顕著な排水を生じるケース}
最後に,より極端な排水が起きる場合の例として$\tilde \mu =\frac{1}{2}\mu_0$で
緩和を行った結果を示す.
図\ref{fig:fig11}は初期状態を含め,4つの時刻における粘土分子の配置
状況を示したものである.
この図には,排水によって生まれたスペースを粘土分子が移動しながら変形して
再配置される様子が示されている.
緩和を終了した時点では,組織構造は当初と大きく異なり,積層した粘土分子の
間に大きな空隙が生じている.
これは,折りたたまれた粘土分子の作る空隙が,緩和が進むにつれて
粘土分子の屈曲が緩むことで少しずつ成長した結果として生じたものである.
ただし,最終的に得られた組織構造に含まれる空隙の一部は,当初存在しないか
非常に小さく目立たないものであったものも含まれる.
図\ref{fig:fig11}-(a)と(d)において(i)から(iii)の番号をつけた
矢印はそのような空隙の例である.
このうち(i)の空隙は,当初三方から積層した粘土分子が押し合うように
して埋めていた領域が開放されたものである.
一方,(ii)と(iii)は粘土分子端部を別の粘土分子が巻き込むようにして配置
されていた箇所である.いずれも,互いに積層する粘土分子に比べて
相互作用が弱く,水分量の減少に伴い,積層した粘土分子のグループが
互いに分離することでスペースが生まれている.
これら,図\ref{fig:fig11}に示したモデルがいずれも同じ乾燥密度を持つことを
踏まえれば,分量とサイズが全く同じ粘土分子の集団であっても,
水分量に応じて極めて多様な組織構造を取りうることを示している.

ここで,緩和過程における温度,応力,エネルギーの推移を見ると,図\ref{fig:fig12}のようになっている.
このケースでは,直応力成分は排水によって急激に減少し,200[ps]程度経過後はほぼ0となっている.
従って,緩和過程の後半に現れる組織構造は,封圧のない状態で生じるものであることが分かる.
なお,エネルギーは,水分量の減少によるエネルギーの低下$\mu \Delta N$が大きく,
全エネルギーの挙動を決める支配的な要因となっている.
ただし,水和エネルギーの変化を見ると,初期に減少した後,増加に転じて飽和している.
これは,初期の脱水が主として水分量の多い粒子から起こす一方で,水分量の少ない粒子は
吸水していることを示すと考えられる.
その後,脱水と吸水が進み水和水層の厚さが平均的に小さくなると,ほぼ均一に脱水が生じる
結果として水和エネルギーが増加に転じている.
このような現象は,粘土含水系の乾燥収縮メカニズムを理解する上で重要な観点を提供するものと期待される.
%--------------------
\begin{figure}[h]
	\begin{center}
	\includegraphics[width=0.8\linewidth]{Figs/fig11.eps} 
	\end{center}
	\caption{
		定積,定温,化学ポテンシャル$\tilde \mu =\frac{1}{2}\mu_0$での緩和挙動.
	} 
	\label{fig:fig11}
\end{figure}
%--------------------
%--------------------
\begin{figure}[h]
	\begin{center}
	\includegraphics[width=1.0\linewidth]{Figs/fig12.eps} 
	\end{center}
	\caption{
		化学ポテンシャル$\tilde \mu =\frac{1}{2}\mu_0$での緩和に伴う
		(a)温度, (b)応力と (c) エネルギーの変化.
	} 
	\label{fig:fig12}
\end{figure}
%--------------------

