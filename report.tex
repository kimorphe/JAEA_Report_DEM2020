\documentclass[11pt,a4j]{jarticle}
\usepackage[dvipdfmx]{graphicx,color}
\usepackage{wrapfig}
\usepackage{amssymb}
\setlength{\topmargin}{-1.5cm}
%\setlength{\textwidth}{15.5cm}
\setlength{\textheight}{25.2cm}
\newlength{\minitwocolumn}
\setlength{\minitwocolumn}{0.5\textwidth}
\addtolength{\minitwocolumn}{-\columnsep}
%\addtolength{\baselineskip}{-0.1\baselineskip}
%
\def\Mmaru#1{{\ooalign{\hfil#1\/\hfil\crcr
\raise.167ex\hbox{\mathhexbox 20D}}}}
%
\begin{document}
\newcommand{\fat}[1]{\mbox{\boldmath $#1$}}
\newcommand{\D}{\partial}
\newcommand{\w}{\omega}
\newcommand{\ga}{\alpha}
\newcommand{\gb}{\beta}
\newcommand{\gx}{\xi}
\newcommand{\gz}{\zeta}
\newcommand{\vhat}[1]{\hat{\fat{#1}}}
\newcommand{\spc}{\vspace{0.7\baselineskip}}
\newcommand{\halfspc}{\vspace{0.3\baselineskip}}
\bibliographystyle{unsrt}
%\pagestyle{empty}
\newcommand{\twofig}[2]
 {
   \begin{figure}[h]
     \begin{minipage}[t]{\minitwocolumn}
         \begin{center}   #1
         \end{center}
     \end{minipage}
         \hspace{\columnsep}
     \begin{minipage}[t]{\minitwocolumn}
         \begin{center} #2
         \end{center}
     \end{minipage}
   \end{figure}
 }
%%%%%%%%%%%%%%%%%%%%%%%%%%%%%%%%%
%\vspace*{\baselineskip}
\begin{center}
{\Large \bf 令和2年度 共同研究報告書}
\end{center}
\vspace{2mm}
\begin{center}
{\LARGE \bf 
メソスケールシミュレーションによる緩衝材の特性評価に関する研究} 
\end{center}
\begin{center}
岡山大学環境生命科学研究科\\
木本和志
\end{center}
\vspace{10mm}
%%%%%%%%%%%%%%%%%%%%%%%%%%%%%%%%%%%%%%%%%%%%%%%%%%%%%%%%%%%%%%%%
\section{はじめに}
\section{はじめに}
ベントナイト緩衝材の変形や物質輸送特性を正確に評価するためには,その
主成分であるモンモリロナイトの膨潤について十分理解することが必要となる.
モンモリロナイトは水分に接触すると,粘土層間の陽イオンに水和する
形で吸水する.このとき,積層した粘土層の層間距離は拡がり,粘土含水系全体
としても膨潤を起こす.膨潤にともない粘土の積層状況や間隙構造は変化する.
そのため,膨潤を伴う水分の浸透は,結果として透水性を変化させる可能性がある.
また粘土含水系に応力を与えて排水させる,すなわち圧排水するためには, 
膨潤圧に抗して粘土を圧縮する必要がある.このように,膨潤は水分浸透や
圧密変形と密接な関係がある.さらに,間隙水分布や間隙ネットワークの構造は,
熱や物質の輸送にも影響するため,膨潤挙動はベントナイトの物性や機能を評価する
上で重要な意味を持つ.

ベントナイトの膨潤や水分浸透の挙動を粘土鉱物スケールで直接的に観測することは難しい.
これは,粘土鉱物や粘土中の間隙が非常に小さく複雑な形状をしているためである.
従って,粘土鉱物スケールでの水和や膨潤の挙動の詳細知るためには,
分子動力学(MD)法を始めとする計算科学的なアプローチをとる必要がある.
ただし,物質を構成する原子一つ一つに自由度を与えて運動方程式を解く全原子MD法は
計算負荷が高く,多数の粘土鉱物と水分子で構成された粘土含水系を扱うことは難しい.
このことを踏まえて本共同研究では,複数の原子や分子を一つの粒子(粗視化粒子)として扱う
粗視化分子動力学法(coarse-grained molecular dynamics method: CGMD法)の開発に
取り組んできた.CGMD法では,モンモリロナイト粘土分子の単位構造と,そこに水和した
間隙水を一つの粗視化粒子で表現してモデルの自由度を削減することで,より多数の
粘土分子の系で凝集や変形の挙動を調べることができる.
昨年度までの研究では,CGMDシミュレーションにより粘土含水系の組織構造モデルを作成して
X線回折パターンを合成し,実験でも観測される観測方向で回折ピークが得られる
こと等を示してきた.また,シミュレーション結果から動径分布関数を評価すれば,
粘土分子の積層挙動や積層数の見積りが得られるなど,粘土含水系組織の形成機構を
理解する上で有用な知見が得られることを示してきた.
しかしながら,これまでのシミュレーションでは間隙水分量は一定としているため,
膨潤や膨潤圧の発生,系が平衡状態にあるときに保持される水分量をシミュレーション
によって求めることはできなかった.
そこで本年度の研究では,粘土含水系に水分の出入りがある場合にも対応できるよう,
CGMD法の拡張を行う.これにより,系外から水分を浸透させることで膨潤や
膨潤圧を発生させることができるようになる.これは,指定された温度や体積,
化学ポテンシャルの下で粘土含水系に保持される水分量を,
CGMD法によって決定することができるようになることを意味する.

本稿の以下では,はじめにCGMD法のモデルとアルゴリズムについて述べる.
その際,水分の移動と水分量の変化に関わるエネルギーの取扱について詳細を示す.
次に,水分量変化を許容した新しいCGMD法によるシミュレーションの結果を示す.
具体的には,乾燥密度が1.6g/cm$^3$程度のモデルを水分量一定の条件でCGMD法に
よって作成し,それを初期モデルとして水分の出入りを許す条件で緩和シミュレーション
を行う.緩和は温度と体積,化学ポテンシャルが一定の条件で,
いくつかの化学ポテンシャルについて行う.
その結果,同一の初期モデルから緩和シミュレーションを開始しても,化学
ポテンシャルの大小によって吸水,排水のいずれも生じうることを示す.
以上の結果を示した後,本年度の研究に関するまとめと今後の課題について述べる.

\section{粗視化MD計算方法}
%%%%%%%%%%%%%%%%%%%%%%%%%%%%%%%%%%%%%%%%%%%%%%%%%%%%%%
\section{粗視化分子動力学(CGMD)法}
\subsection{運動方程式}
本研究のCGMD法では、粘土分子の単位構造とそこに水和された水分子を一つの
粗視化粒子として扱う.いま,$n$個の粗視化粒子があるものとし,そのうち
$i$番目のものの質量を$m_i$,速度ベクトルを$\fat{v}_i=\dot{\fat{x}}_i$,
位置ベクトルを$\fat{x}_i$とすれば,粒子$i$の運動方程式は
\begin{equation}
	m_i \dot {\fat{v}}_i =\fat{f}_i ,\ \ 
	\dot{\fat{x}}_i = \fat{v}_i, \ \ ( i =1,2,\cdots n )
	\label{eqn:eq_mot}
\end{equation}
と表される.ここで,$\fat{f}_i$は粒子$i$に作用する力のベクトルを意味する.
なお,素視化粒子の質量は,粘土分子の単位構造の質量から
\begin{equation}
	m=2.468\times 10 ^{-24}[{\rm kg}]
	\label{eqn:mass}
\end{equation}
としておく.粒子に作用する力のベクトル$\fat{f}_i$は
分子間力$\fat{f}_i^{U}$と分子内力$\fat{f}_i^{K}$の和として
\begin{equation}
	\fat{f}_i=\fat{f}_i^U+\fat{f}_i^K
	\label{eqn:f_tot}
\end{equation}
と書くことができる.粒子に作用するこれらの力の与え方は
既報(2018年度共同研究報告書)に述べた通りで,水和水の多寡は
分子間力$f_i^U$に含まれるパラメータ$\sigma$で表現される.
本年度の研究では,水和水量の増減による組織構造の変化や
膨潤圧の発生に関するシミュレーションを行う.
そのための方法を示すために,ここででは水分の挙動に関係する
$\fat{f}_i^U$の詳細を述べる.その後,粘土含水系内での水分配置や
水分の総量を更新する手法について説明する.\\

素視化粒子間に作用する分子間力は,レナード-ジョーンズ(LJ)型の
ペアポテンシャル:
\begin{equation}
	U(\fat{x}_i,\fat{x}_j; \sigma) 
	= 4 \varepsilon 
	\left\{ 
	\left(\frac{\sigma}{r_{ij}}\right)^{12}
	-
	\left(\frac{\sigma}{r_{ij}}\right)^6
	\right\}, \ \ \left( r_{ij}=\left| \fat{x}_i-\fat{x}_j\right| \right)
	\label{eqn:LJ}
\end{equation}
を用いて次のように与える.
\begin{equation}
	\fat{f}_i^U(\fat{x}_i)
	=
	-\nabla_{x_i} 
	\left\{ 
		\sum_{j=1, \, j \neq i}^n U\left(\fat{x}_i,\,\fat{x}_j; \sigma \right)
	\right\}
	\label{eqn:fiU}
\end{equation}
ここに$\fat{x}_i$と$\fat{x}_j$は,現在時刻における粒子$i$と粒子$j$の位置ベクトルを,
$\varepsilon$と$\sigma$はLJポテンシャルのパラメータを,$\nabla_{x_i}=\frac{\partial}{\partial \fat{x}_i}$
はLJポテンシャルの 第一引数$\fat{x}_i$に関する勾配を表す.
LJポテンシャルの特性距離を与えるパラメータ$\sigma$は,素視化粒子間の接近限界,すなわち,粘土分子が積層
した際の層間距離を定める.従って,これを当該粒子に水和した水分の量を表す変数と考えることができる.
なお,モンモリロナイトの膨潤状態と$\sigma$の対応は,X線回折試験によって得られた粘土層間距離から
概ね表\ref{tbl:tbl_sig}のようになる.$\sigma$の値は計算途上で$0.9$nmを下限値として適宜更新される.
一方,$\varepsilon$は,粗視化粒子間相互作用の強さを決定するパラメータであり、
事前に行った全原子MDシミュレーションの結果を参考に
\begin{equation}
	\varepsilon=1.0\times 10^{-19}[{\rm Nm}]
	\label{eqn:LJ_eps}
\end{equation}
と与え,こちらは定数として扱う.
\begin{table}[h]
	\begin{center}
	\caption{分子間相互作用ポテンシャルにおける特性距離と膨潤状態の対応.}
	\vspace{3mm}
	\begin{tabular}{c||c|c|c|c|c}
		膨潤状態 & 0層 & 1層 & 2層 & 3層 & $\cdots$\\
		\hline
		特性距離$\sigma$[{\rm nm}]& 0.9 & 1.2 & 1.5 & 1.8 & $\cdots$ \\
		\hline
		水和水層厚$s^\pm_i$[{\rm nm}] & 0.0 & 0.15 & 0.30 & 0.45 & $\cdots$
	\end{tabular}
	\label{tbl:tbl_sig}
	\end{center}
\end{table}
\subsection{特性距離の設定}
粘土分子表面に水和した水分の移動を許容する場合,空間的に水分量の偏りが生じる.
そのような状況を計算上表現するためには,水和水量を与えるパラメータ$\sigma$を
粒子ペア毎に設定する必要がある.また,粘土分子の一方の面と他方の面に水和した
水分量が異なる場合,粘土分子の上下いずれの側と相互作用するか区別して
ペアポテンシャルを評価しなければならない.このことを考慮し,本研究では以下の
手順で$\sigma$の値を設定する.

図\ref{fig:fig13}を参照し,粒子$i$と粒子$j$に関するペアポテンシャル(\ref{eqn:LJ})の
特性距離$\sigma$について考える.いま,$\fat{x}_i$にある粒子$i$が属する粘土分子の一方の面を
$S^+$,他方の面を$S^-$とする.また,粒子位置$\fat{x}_i$における$S^+$の法線ベクトルを$\fat{n}_i$,
$S^-$の法線ベクトルを$-\fat{n}$とする.ここで,粒子$i$の$S^+$側に水和した水和水層の厚さを
$s_i^+$,$S^-$側の厚さを$s_i^-$と書く.
このとき,粒子$j$が$\fat{n}^+$の側にある場合は$S^+$と,$-\fat{n}$の側にある場合は
$S^-$と相互作用すると考えることができる.
従って,該当する側の面を水和水層厚を$s_i$とすれば,相互作用する粒子の位置
に応じて$s_i$を
\begin{equation}
	s_i=\left\{
	\begin{array}{cc}
		s_i^+ & (h(\fat{x}_i,\fat{x}_j) \ge 0)\\
		s_i^- & (h(\fat{x}_j,\fat{x}_i) <0)
	\end{array}
	\right.
	\label{eqn:si_switch}
\end{equation}
\begin{equation}
	h(\fat{x}_i,\fat{x}_j) =\fat{n}_i\cdot \left(\fat{x}_j-\fat{x}_i\right)
	\label{eqn:h_sgn}
\end{equation}
で選択することができる.そこで,粒子$i$と粒子$j$の間のペアポテンシャルの特性距離$\sigma$を
\begin{equation}
	\sigma=s_i+s_j+\sigma_0
	\label{eqn:sig_ij}
\end{equation}
で与える.ただし,$\sigma_0(=0.9)$[nm]は無水時(0層膨潤状態)の層間距離を表す.
なお,図\ref{fig:fig13}は,粒子位置$\fat{x}_j$が前者の場合について図示したものである.
このようにすることで,粘土分子の表裏面を区別し,位置に応じて異なる水和水の分布を表現することができる.
なお,$s_i=s_j=s$のとき,水和水層の厚さ$s$と膨潤状態の対応は表\ref{tbl:tbl_sig}のようになる.
%--------------------
\begin{figure}[h]
	\begin{center}
	\includegraphics[width=0.5\linewidth]{Figs/fig13.eps} 
	\end{center}
	\caption{
		粘土分子表面の水和水層の厚さ$s_i^\pm $.
		$h$は,粒子間相互作用の計算において考慮すべき
		面($S^+$あるいは$S^-$)を判定するための符号付き距離を表す.
	} 
	\label{fig:fig13}
\end{figure}
%--------------------
\subsection{時間ステッピング}
CGMD法では粒子位置$\fat{x}_i$と速度$\fat{v}_i$に加え,水和層の厚さ$s_i^\pm$が
系の時間発展に伴い更新の対象となる.以下,これらの未知量を
\begin{equation}
	\fat{V}=\left\{ \fat{v}_1,\, \fat{v}_2,\, \dots \fat{v}_n \right\}
	\label{eqn:defV}
\end{equation}
\begin{equation}
	\fat{X}=\left\{ \fat{x}_1,\, \fat{x}_2,\, \dots \fat{x}_n \right\}
	\label{eqn:defV}
\end{equation}
\begin{equation}
	\fat{\Sigma}=\left\{ \fat{s}_1^\pm,\, \fat{s}_2^\pm,\, \dots \fat{s}_n^\pm \right\}
	\label{eqn:defV}
\end{equation}
と表し,各々の時間ステッピングにともなう増分を$\Delta \fat{V},\Delta \fat{X}$,
および$\Delta \fat{\Sigma}$と書く.これらのうち、位置と速度の更新は運動方程式(\ref{eqn:eq_mot})
を差分法で離散化して時間積分することによって行う.離散化のための差分スキームにはleapfrog法による
中央差分を用いる.なお,$\fat{V}$と$\fat{X}$の更新時には$\fat{\Sigma}$を一定としておく.すなわち,
\begin{equation}
	\left( \fat{V},\,\fat{X} \right)
	\rightarrow 
	\left( \fat{V},\,\fat{X} \right)
	+
	\left. \left( \Delta \fat{V},\, \Delta \fat{X} \right) \right|_{\fat{\Sigma}}
	\label{eqn:vx_update}
\end{equation}
とする. 一方,水分分布$\fat{\Sigma}$の更新にはモンテカルロ法を用い,その際$(\fat{V},\fat{X})$は
一定としておく.すなわち,水分量の状態更新は
\begin{equation}
	\fat{\Sigma} \rightarrow \fat{\Sigma} + \left. \Delta \fat{\Sigma} \right|_{\fat{V},\fat{X}}
	\label{eqn:s_update}
\end{equation}
と表すことができる.
$\fat{\Sigma}$の更新に関するモンテカルロ法では,
疑似乱数を用いて粗視化粒子系が持つ水分に関するエネルギー$U(\fat{V},\fat{X},\fat{\Sigma})$が
低減する方向へ繰返し状態更新を行う.ここでは水分の移動に関して二種類の状態更新を行う.
1つ目の状態更新では,粗視化粒子のペアを無作為に選択してこれらの粒子間で水分の授受を行うべきかを判定する.
その際,水分授受が発生した仮定の元で,粗視化粒子系の全エネルギーの増減$\Delta U$を計算する.
$\Delta U<0$の場合には実際に(\ref{eqn:s_update})のように水分の状態を更新する.
この方法では系内の水分量は一定に保たれる.
2つ目の状態更新では,一つの粗視化粒子を無作為に選択してその水分量が所定量だけ増加あるいは減少したと仮定する.
その仮定の元で生じるエネルギー変化$\Delta U$を計算し,$\Delta U<0$の場合に実際に当該粒子が保持する
水分量を変化させる.このような状態更新を行うことで,系内に含まれる水分の総量を変化させることができる.
次節ではモンテカルロ法におけるエネルギー$U$の具体的な評価方法を述べる.
\subsection{水分量と水分分布に関するエネルギー}
水分の分布や量に関する全エネルギーを$E$とし,これを
\begin{equation}
	E=U_{LJ} +U_{hyd} + U_N
	\label{eqn:}
\end{equation}
と表す.ここで$U_{LJ}$は粗視化粒子間相互作用の,$U_{hyd}$は水分子と粘土分子の水和による
相互作用の,$U_N$は系内に含まれる水分子数の増減に関するエネルギーをそれぞれ表す.
このうち$U_{LJ}$は式(\ref{eqn:LJ})を用い,
\begin{equation}
	U_{LJ}=\sum_{i\neq j} U(\fat{x}_i,\fat{x}_j;\sigma)
	\label{eqn:}
\end{equation}
で与えられ,運動方程式の積分でも考慮されるものである.
一方、$U_{hyd}$は粗視化粒子内のエネルギーであるため、運動方程式や粒子間相互作用
ポテンシャルには含まれていない.そこで,$U_{hyd}$は後述する方法により別途モデル化して与える.
残る$U_N$は,水分量の増減によって変化するエネルギーで、
\begin{equation}
	U_N=\mu N
	\label{eqn:U_N}
\end{equation}
で与える.ここに,$\mu$は水分に関する化学ポテンシャルを, $N$は水分子の総数を表す.
なお,実際のCGMDシミュレーションでは,水分子数ではなく水和水の層の厚さ$s(=s^{+}$あるいは$s=s^{-})$を変化させる. 
そのため,$\Delta U_N$の計算にあたり,
\begin{equation}
	\Delta U_N =\tilde \mu \Delta s
	\label{eqn:}
\end{equation}
として,$\mu$ではなく
\begin{equation}
	\tilde \mu =\mu \frac{\Delta N}{\Delta s}
	\label{eqn:}
\end{equation}
を設定して計算を行う.
\subsection{水和エネルギーモデル}
粒子$i$の$S^\beta$面側の水分に関する水和エネルギーを$u^\beta_i(\beta=+,-)$とする.
このとき系全体の水和エネルギーは
\begin{equation}
	U_{hyd}=\sum_{i=1}^n\sum_{\beta=+,-} u_i^{\beta}
	\label{eqn:}
\end{equation}
で与えられる.$u^+_i$と$u^-_i$を与える関数は,水分量$s$には依存するが,
粗視化粒子$(i)$や面の向き($\beta$)に依らないと考えてよい.そこで,
\begin{equation}
	u^\beta_i= u(s^\beta), \ \ (s^\beta>0,\, \beta=+,-)
	\label{eqn:}
\end{equation}
として,$u$を水和水層の厚さ$s$の関数としてモデル化する.
$u(s)$は$s$に対して大局的には減少する有界な関数と考えられ,
$u(0)=0$としてよい.そのような挙動を示す簡単な関数として,
ここでは次のような指数関数を用いる.
\begin{equation}
	u(s)=u_{\infty} \left(1-e^{-\gamma s} \right)
	\label{eqn:u_s}
\end{equation}
ここに,$u_{\infty}$は$s\rightarrow \infty$(無限膨潤)での水和エネルギーを表す.
一方,$\gamma$は
\begin{equation}
	u'(s)=\frac{du}{ds}=\gamma u_{\infty} e^{-\gamma s}
	\label{eqn:ud_s}
\end{equation}
だから,水和エネルギーの$s$に対する変化率を決めるパラメータとなる.
ここでは,$u'(s)$の半減距離$s=s_b$を定め,
\begin{equation}
	\frac{u'\left(s_b\right)}{u'(0)}=e^{-\gamma s_b}=\frac{1}{2} 
	\ \ \Rightarrow \ \
	\gamma=\frac{\log 2}{s_b}
	\label{eqn:}
\end{equation}
から$\gamma$の値を決定した.このようにモデル化した$u(s)$と$u'(s)$は,
図\ref{fig:fig0}に示すような形状を示す.
なお,図\ref{fig:fig0}のグラフでは,横軸が半減距離$s_b$で無次元化されている
ため,$u(s),u'(s)$とも$s/s_b=1$において$u(0),u'(0)$から値が半減している
ことが確認できる.
%--------------------
\begin{figure}[h]
	\begin{center}
	\includegraphics[width=1.0\linewidth]{Figs/fig0.eps} 
	\end{center}
	\caption{
		水和エネルギー$u$とその導関数$u'$.
		$s$は水和水層の厚さ,$s_b$は$-u'$の半減距離,
		$\gamma$は減衰率($e^{-\gamma s_b}=1/2$)を表す.
	} 
	\label{fig:fig0}
\end{figure}
%--------------------

\section{計算例}
%\section{水分量変化を許容したCGMDシミュレーション}
本節では定温定積かつ化学ポテンシャル一定の元で行った粘土含水系の緩和シミュレーション結果を示す.
緩和シミュレーションの初期モデルには,前節で示した圧縮凝集と冷却過程を経て得られた組織構造モデルを用いる.
ここでは,ユニットセルのサイズを初期モデルの状態で固定,温度を300[K]に設定し,
一定の化学ポテンシャル$\tilde \mu$のもとで1[ns]間,CGMDシミュレーションを行う.
\subsection{組織構造の変化}
図\ref{fig:fig4}は,$\tilde \mu=0$として行った緩和シミュレーションで得られた
組織構造を示したものである.この場合,$\tilde \mu=0$であることから,
水分量の増減によるエネルギーの変化$\mu \Delta N$が無い.そのため,
水和エネルギー$U_{hyd}$の減少がポテンシャルエネルギー$U_{LJ}$の増加を上回る限り,
系内の水分量は増加することができる.
図\ref{fig:fig4}に示した結果では,積層した粘土層間の距離が緩和の結果初期状態より
大きくなり,同時に層間距離が均一化する様子が見られる.
例えば,図\ref{fig:fig4}-(a)において(i)と(ii)の破線で囲んだ箇所では,当初,層間距離が
にばらつきが見られる.これらの箇所は吸水により最終的に層間距離がほぼ一様になっている.
また,粘土分子が折りたたまれて出来た空隙部分も水分の増加に伴い,その大きさは次第に
小さくなっている.図\ref{fig:fig4}-(a)に示した矢印はそのような空隙の一例を示したものである.
%
これに対して図\ref{fig:fig5}は,化学ポテンシャルを
\begin{equation}
	\frac{\tilde \mu}{\mu _0}= \frac{1}{5}
	, \ \ 
	\left( 
		\mu_0:=u'(0)=\gamma u_\infty
	\right)
	\label{eqn:mu02}
\end{equation}
とした場合の結果を示したものである.式(\ref{eqn:mu02})は,水和エネルギー勾配$u'(s)$の最大値
$u'(0)=\gamma u_{\infty}$を基準$\mu_0$として化学ポテンシャルを与えることを意図したものである.
これを$\mu=\frac{1}{5}$とした理由は,水分量の変動によるエネルギーの増減$\mu \Delta N$が,
水和エネルギーの変動$\Delta U_{hyd}$を凌駕することなく,適度な排水を生じさせることを意図
したためである.
実際,図\ref{fig:fig5}の結果では,一部の水分が排出されることで空隙が増加する方向へ組織構造が
変化している.詳しくみれば,水分量の減少によって生まれたスペースを使って粘土分子が若干移動することで,
屈曲が解消される箇所(図\ref{fig:fig5}-(c)の(iii)で示した箇所)や,
空洞が消失や統合する様子が示されている(それぞれ,図\ref{fig:fig5}-(b)の(i)と(ii)の箇所).
ただし,排水量はあまり大きくないために,粘土分子の相対的な配置や分子の屈曲状況の変化は
劇的なものではない.
%--------------------
\begin{figure}[h]
	\begin{center}
	\includegraphics[width=1.0\linewidth]{Figs/fig4.eps} 
	\end{center}
	\caption{
		定積,定温,化学ポテンシャル一定での緩和挙動($\tilde \mu =0$の場合).
	} 
	\label{fig:fig4}
\end{figure}
%--------------------
%--------------------
\begin{figure}[h]
	\begin{center}
	\includegraphics[width=1.0\linewidth]{Figs/fig5.eps} 
	\end{center}
	\caption{
		定積,定温,化学ポテンシャル一定での緩和挙動($\tilde \mu =\frac{1}{5}\mu_0 $の場合).
	} 
	\label{fig:fig5}
\end{figure}
%--------------------
\subsection{水分分布の変化}
次に,緩和前後での水分分布状況について調べる.
図\ref{fig:fig6}は,各粗視化粒子が有する水分量を,緩和前(初期状態)と
緩和後($\tilde \mu=0$と$\tilde \mu =\frac{\mu_0}{5}$)の場合について示したも
のである.横軸は粗視化粒子番号$i(=1,\dots N)$を,縦軸は水和水層の厚さ($s_i^+,\, s^i_-$)を表す.
一つの粘土分子を構成する粗視化粒子には,連続する粒子番号が与えられている.
そこでこの図では,粒子が属する分子が異なるものになるたびに,異なる色を用いて
水和水層の厚さをプロットしている.従って,同一色の線で描かれた区間が粘土分子
毎の水分量を表している.なお$s^+_i$は実線で,$s^-_i$は破線で表示しているが,
両者はほとんどの場合,非常に近い値となるためグラフ上ではほとんど区別がつかなくなっている.
図\ref{fig:fig7}は同じ水分量のデータを$s^\pm_i$のヒストグラムとして表示したもので,
分子毎の水分量に関する情報は失われるが,水和水層厚さの分布を見るためのものである.
これら図\ref{fig:fig6}と図\ref{fig:fig7}から明らかな通り,水分量は当初大きく
偏った状態にある.このように偏在した水分が,緩和後は概ね均等に配分さている.
特に,化学ポテンシャルの値が相対的に大きな$\tilde \mu= \frac{1}{5}\mu_0$の場合,
水和水層厚は,$s^{\pm}_i\simeq 0.23$[nm]程度の非常に狭い範囲に集中して
分布していることが図\ref{fig:fig7}-(c)に示されている.
なお,これはおよそ1.5層膨潤程度の状態にあることを意味する.
一方,$\tilde \mu=0$の場合,$s^\pm_i$の平均は0.3[nm]で2層膨潤程度となるが,
分布幅は$\tilde \mu =\frac{1}{5}\mu_0$のケースに比べて広く,
いくつかの粘土分子で平均値よりも明らかに大きな水和水層厚となっている.
これは,吸水に伴い粘土分子の相対位置が制限され,相対的に大きな空隙が
残されたまま水分が充填されるためと考えられる.
また,$s^\pm_i$の下限値は0.26[nm]で,ヒストグラムは非対称な形状を示し,
これは水和エネルギー$u(s)$の分布に起因したもので,
物理的には水分量が少ないとき,非常に強い水和が起きることに対応している.
%--------------------
\begin{figure}[h]
	\begin{center}
	\includegraphics[width=0.8\linewidth]{Figs/fig6.eps} 
	\end{center}
	\caption{
		各粗視化粒子における水和水層の厚さ$s_i^+,s_i^-$.
		(a)緩和開始時と,緩和後(b)$\tilde \mu=0$, (c)$\tilde \mu =\frac{1}{5}\mu_0$
		に対する結果.
	} 
	\label{fig:fig6}
\end{figure}
%--------------------
%--------------------
\begin{figure}[h]
	\begin{center}
	\includegraphics[width=0.6\linewidth]{Figs/fig7.eps} 
	\end{center}
	\caption{
		各粗視化粒子における水和水層の厚さ$s_i^+,s_i^-$のヒストグラム.
		(a)緩和開始時と,緩和後(b)$\tilde \mu=0$, (c)$\tilde \mu =\frac{1}{5}\mu_0$
		に対する結果.
	} 
	\label{fig:fig7}
\end{figure}
%--------------------
\subsection{応力とエネルギーの推移}
ここではユニットセル体積を一定としているため,緩和により吸水が起きる場合には
膨潤圧が発生し,排水が起きる場合には応力が減じる.図\ref{fig:fig9}と図\ref{fig:fig10}は
このことを$\tilde mu=0$と$\frac{1}{5}\mu_0$の場合についてそれぞれ示したもので,
応力の推移と併せて温度とエネルギーの変化も示されている.
温度は300[K]に設定されているが,必ずゆらぎを伴う.
特に,水分の移動により粗視化粒子の運動状況が変化するため,ゆらぎの幅は
水分分布を固定した場合よりも大きくなると予想されるが,図\ref{fig:fig9},図\ref{fig:fig10}の
結果とも,300[K]の周辺で$\pm$15[K]程度のゆらぎにとどまっていることが分かる.
応力は,せん断力成分$\sigma_{12}$は零付近でほぼ変が無い.
一方,直応力成分$\sigma_{11},\sigma_{22}$は互いに等しく等方的な圧力が生じており,
これが緩和により変動する様子が示されている.
$\tilde \mu=0$のときには,吸水によって直応力は増加した後一定値に至っている.
この間,ポテンシャルエネルギー$U_{LJ}$は増加しているが,水分の増加による
水和エネルギー$\Delta U_{hyd}$の減少分が$U_{LJ}$を大きく上回ることで,
系全体のエネルギーが単調に減少して下限値に近づくことが,図\ref{fig:fig9}-(c)に
示されている.このグラフでは,分子間相互作用のポテンシャル$U_{LJ}$と,
水和エネルギーの初期状態からの変化$\Delta U_{hyd}(t)=U_{hyd}(t)-U_{hyd}(0)$,
水分量の変化によるエネルギの変化
\begin{equation}
	\mu \Delta N= \mu \left(N(t)-N(0) \right) = \sum_i^n \int_0^t \tilde \mu 
	\frac{d\sigma_i}{dt} dt
	\label{eqn:mu_dN}
\end{equation}
と,それらの合計を示している.
ここに,$N(t)$は時刻$t$における水分子の総数を意味する.
$\tilde \mu=0$であるため,図\ref{fig:fig9}-(c)で$\mu \Delta N$に変化はない.
%
一方,$\tilde \mu =1/2\mu_0$のときには,直応力は排水によってやや振動しながら低下して
一定値に漸近する.エネルギーに関しては,水分量の減少によるエネルギ−変化$\mu \Delta N$と
水和ネルギー$\Delta U_{hyd}$とも低下し,ポテンシャルエネルギーはわずかに増加している.
$\mu \Delta N$と$\Delta U_{hyd}$がともに減少することは,
水和水層の厚さが大きな箇所からの排水量が,層厚の小さな箇所への給水量を上回ることを
意味している.
% 応力の変化=(運動エネルギー−分子間力)の変化
%--------------------
\begin{figure}[h]
	\begin{center}
	\includegraphics[width=1.0\linewidth]{Figs/fig9.eps} 
	\end{center}
	\caption{
		化学ポテンシャル一定での緩和による
		(a)温度, (b)応力と(c)エネルギーの変化 ($\tilde \mu=0$).
	} 
	\label{fig:fig9}
\end{figure}
%--------------------
%--------------------
\begin{figure}[h]
	\begin{center}
	\includegraphics[width=1.0\linewidth]{Figs/fig10.eps} 
	\end{center}
	\caption{
		化学ポテンシャル一定での緩和による(a)温度, (b)応力と (c) エネルギーの変化
		($\tilde \mu =\frac{1}{5}\mu_0).$
	} 
	\label{fig:fig10}
\end{figure}
%--------------------
\subsection{より顕著な排水を生じるケース}
最後に,より極端な排水が起きる場合の例として$\tilde \mu =\frac{1}{2}\mu_0$で
緩和を行った結果を示す.
を図\ref{fig:fig11}は,初期状態を含め,4つの時刻における粘土分子の配置
状況を示したものである.
この図には,排水によって生まれたスペースを粘土分子が移動しながら変形して
再配置される様子が示されている.
緩和を終了した時点では,組織構造は当初と大きく異なり,積層した粘土分子の
間に大きな空隙が生じている.
これは,折りたたまれた粘土分子の作る空隙が,緩和が進むにつれて
粘土分子の屈曲が緩むことで少しずつ成長した結果として生じたものである.
ただし,最終的に得られた組織構造に含まれる空隙の一部は,当初存在しないか
非常に小さく目立たないものであったものも含まれる.
図\ref{fig:fig11}-(a)と(d)において(i)から(iii)の番号をつけた
矢印はそのような空隙の例である.
このうち(i)の空隙は,当初三方から積層した粘土分子が押し合うように
して埋めていた領域が開放されたものである.
一方,(ii)と(iii)は粘土分子端部を別の粘土分子が巻き込むようにして配置
されていた箇所である.いずれも,互いに積層する粘土分子に比べて
相互作用が弱く,水分量の減少に伴い,積層した粘土分子のグループが
互いに分離することでスペースが生まれている.
これら,図\ref{fig:fig11}に示したモデルがいずれも同じ乾燥密度を持つことを
踏まえれば,分量とサイズが全く同じ粘土分子の集団であっても,
水分量に応じて極めて多様な組織構造を取りうることを示している.

ここで,緩和過程における温度,応力,エネルギーの推移を見ると,図\ref{fig:fig12}のようになっている.
このケースでは,直応力成分は排水によって急激に減少し,200[ps]程度経過後はほぼ零となっている.
従って,緩和過程の後半に現れる組織構造は,封圧のない状態で生じるものであることが分かる.
なお,エネルギーは,水分量の減少によるエネルギーの低下$\mu \Delta N$が大きく,
全エネルギーの挙動を決める支配的な要因となっている.
ただし,水和エネルギーの変化を見ると,初期に減少した後,増加に転じて飽和している.
これは,初期の脱水が主として水分量の多い粒子から起こす一方で,水分量の少ない粒子は
吸水していることを示すと考えられる.
その後,脱水と吸水が進み水和水層の厚さが平均的に小さくなると,ほぼ均一に脱水が生じる
結果として水和エネルギーが増加に転じている.
このような現象は,粘土含水系の乾燥収縮メカニズムを理解する上で重要な観点を提供するものと期待される.
%--------------------
\begin{figure}[h]
	\begin{center}
	\includegraphics[width=0.8\linewidth]{Figs/fig11.eps} 
	\end{center}
	\caption{
		定積,定温,化学ポテンシャル一定での緩和挙動($\tilde \mu =\frac{1}{2}\mu_0$の場合).
	} 
	\label{fig:fig11}
\end{figure}
%--------------------
%--------------------
\begin{figure}[h]
	\begin{center}
	\includegraphics[width=1.0\linewidth]{Figs/fig12.eps} 
	\end{center}
	\caption{
		化学ポテンシャル一定での緩和により(a)温度, (b)応力と (c) エネルギーの変化
		($\tilde \mu =\frac{1}{2}\mu_0).$
	} 
	\label{fig:fig12}
\end{figure}
%--------------------


\section{まとめと今後の課題}
粘土含水系の組織構造を特徴づける分布として,X線回折(XRD)パターン,動径分布関数(RDF),局所密度分布と
配向確率密度の4種類の分布をメソMD計算結果から評価する方法を示した.その結果,メソMD
結果から得たXRDパターンは,実際に実験でも観測されるピークが現れることを示した.
また,RDFのピーク数をカウントすることによって積層数の見積が得られることを示し,こちらについても
実験結果として観測される程度の値となることが分かった.以上の結果は,これまでに開発してきた
メソMD解析法の妥当性を裏付けるものと言える.
局所密度分布と配向性については,メソMD計算の結果を詳しく調べ,粘土含水系の圧縮凝集挙動や
物質輸送特性の解明において,有用なツールになると考えられる.
特に,メソ間隙とナノ間隙を区別することは局所密度のデータを用いて実現可能と考えられ,
今後マルチスケール解析を行う際には,ナノ構造を均質化してメソ多孔質体モデルを作成する
際に直接利用できると考えられる.また,配向性に関しては,今回の結果からは明確な発見は
なかったものの,局所的な配向性の変化は,圧縮性や透水,拡散等の物質輸送現象の異方性について
理解する上で重要な指標に成りうることが期待できる.
今後は,実験データとの比較により,メソMD結果の妥当性を検証することが課題となる.
また,ナノメートルからマイクロメートルスケールの透水問題は,緩衝材の膨潤や物質輸送挙動
を理解する上で重要であることから,水分浸透問題へ,メソMD法の拡張を図ることも重要な課題
の一つとなる.その結果,マクロな輸送係数や変形特性を分子スケールでの計算から定量的に評価
することが可能となれば,実験や直接観測が難しい種々の物理化学的条件においてマクロ物性を
予測することで,連続体ベースでの熱や水,力学についての解析の詳細化や高度化にも貢献できると考えられる.
\begin{thebibliography}{99}
\bibitem{Warren}
	B.E. Warren, "X-ray diffraction", Dover Publication, New York, 1990.
\bibitem{Koide}
	小出昭一郎,"物理現象のフーリエ解析", ちくま学芸文庫, 2018.
\end{thebibliography}

\end{document}
%%%%%%%%%%%%%%%%%%%%%%%%%%%%%%%%%%%%%%%%%%%%%%%%%%%%%%

