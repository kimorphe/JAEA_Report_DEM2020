\documentclass[11pt,a4j]{jarticle}
\usepackage[dvipdfmx]{graphicx,color}
%\usepackage{showkeys}
\usepackage{wrapfig}
\usepackage{amssymb}
\setlength{\topmargin}{-1.5cm}
%\setlength{\textwidth}{15.5cm}
\setlength{\textheight}{25.2cm}
\newlength{\minitwocolumn}
\setlength{\minitwocolumn}{0.5\textwidth}
\addtolength{\minitwocolumn}{-\columnsep}
%\addtolength{\baselineskip}{-0.1\baselineskip}
%
\def\Mmaru#1{{\ooalign{\hfil#1\/\hfil\crcr
\raise.167ex\hbox{\mathhexbox 20D}}}}
%
\begin{document}
\newcommand{\fat}[1]{\mbox{\boldmath $#1$}}
\newcommand{\D}{\partial}
\newcommand{\w}{\omega}
\newcommand{\ga}{\alpha}
\newcommand{\gb}{\beta}
\newcommand{\gx}{\xi}
\newcommand{\gz}{\zeta}
\newcommand{\vhat}[1]{\hat{\fat{#1}}}
\newcommand{\spc}{\vspace{0.7\baselineskip}}
\newcommand{\halfspc}{\vspace{0.3\baselineskip}}
\bibliographystyle{unsrt}
%\pagestyle{empty}
\newcommand{\twofig}[2]
 {
   \begin{figure}[h]
     \begin{minipage}[t]{\minitwocolumn}
         \begin{center}   #1
         \end{center}
     \end{minipage}
         \hspace{\columnsep}
     \begin{minipage}[t]{\minitwocolumn}
         \begin{center} #2
         \end{center}
     \end{minipage}
   \end{figure}
 }
%%%%%%%%%%%%%%%%%%%%%%%%%%%%%%%%%
%\vspace*{\baselineskip}
\begin{center}
{\Large \bf 令和2年度 共同研究報告書}
\end{center}
\vspace{2mm}
\begin{center}
{\LARGE \bf 
メソスケールシミュレーションによる緩衝材の特性評価に関する研究} 
\end{center}
\begin{center}
岡山大学環境生命科学研究科\\
木本和志
\end{center}
\vspace{10mm}
%%%%%%%%%%%%%%%%%%%%%%%%%%%%%%%%%%%%%%%%%%%%%%%%%%%%%%%%%%%%%%%%
\section{はじめに}
ベントナイト緩衝材の変形や物質輸送特性を正確に評価するためには,その
主成分であるモンモリロナイトの膨潤について十分理解することが必要となる.
モンモリロナイトは水分に接触すると,粘土層間の陽イオンに水和する
形で吸水する.このとき,積層した粘土層の層間距離は拡がり,粘土含水系全体
としても膨潤を起こす.膨潤にともない粘土の積層状況や間隙構造は変化する.
そのため,膨潤を伴う水分の浸透は,結果として透水性を変化させる可能性がある.
また粘土含水系に応力を与えて排水させる,すなわち圧排水するためには, 
膨潤圧に抗して粘土を圧縮する必要がある.このように,膨潤は水分浸透や
圧密変形と密接な関係がある.さらに,間隙水分布や間隙ネットワークの構造は,
熱や物質の輸送にも影響するため,膨潤挙動はベントナイトの物性や機能を評価する
上で重要な意味を持つ.

ベントナイトの膨潤や水分浸透の挙動を粘土鉱物スケールで直接的に観測することは難しい.
これは,粘土鉱物や粘土中の間隙が非常に小さく複雑な形状をしているためである.
従って,粘土鉱物スケールでの水和や膨潤の挙動の詳細知るためには,
分子動力学(MD)法を始めとする計算科学的なアプローチをとる必要がある.
ただし,物質を構成する原子一つ一つに自由度を与えて運動方程式を解く全原子MD法は
計算負荷が高く,多数の粘土鉱物と水分子で構成された粘土含水系を扱うことは難しい.
このことを踏まえて本共同研究では,複数の原子や分子を一つの粒子(粗視化粒子)として扱う
粗視化分子動力学法(coarse-grained molecular dynamics method: CGMD法)の開発に
取り組んできた.CGMD法では,モンモリロナイト粘土分子の単位構造と,そこに水和した
間隙水を一つの粗視化粒子で表現してモデルの自由度を削減することで,より多数の
粘土分子の系で凝集や変形の挙動を調べることができる.
昨年度までの研究では,CGMDシミュレーションにより粘土含水系の組織構造モデルを作成して
X線回折パターンを合成し,実験でも観測される観測方向で回折ピークが得られる
こと等を示してきた.また,シミュレーション結果から動径分布関数を評価すれば,
粘土分子の積層挙動や積層数の見積りが得られるなど,粘土含水系組織の形成機構を
理解する上で有用な知見が得られることを示してきた.
しかしながら,これまでのシミュレーションでは間隙水分量は一定としているため,
膨潤や膨潤圧の発生,系が平衡状態にあるときに保持される水分量をシミュレーション
によって求めることはできなかった.
そこで本年度の研究では,粘土含水系に水分の出入りがある場合にも対応できるよう,
CGMD法の拡張を行う.これにより,系外から水分を浸透させることで膨潤や
膨潤圧を発生させることができるようになる.これは,指定された温度や体積,
化学ポテンシャルの下で粘土含水系に保持される水分量を,
CGMD法によって決定することができるようになることを意味する.

本稿の以下では,はじめにCGMD法のモデルとアルゴリズムについて述べる.
その際,水分の移動と水分量の変化に関わるエネルギーの取扱について詳細を示す.
次に,水分量変化を許容した新しいCGMD法によるシミュレーションの結果を示す.
具体的には,乾燥密度が1.6g/cm$^3$程度のモデルを水分量一定の条件でCGMD法に
よって作成し,それを初期モデルとして水分の出入りを許す条件で緩和シミュレーション
を行う.緩和は温度と体積,化学ポテンシャルが一定の条件で,
いくつかの化学ポテンシャルについて行う.
その結果,同一の初期モデルから緩和シミュレーションを開始しても,化学
ポテンシャルの大小によって吸水,排水のいずれも生じうることを示す.
以上の結果を示した後,本年度の研究に関するまとめと今後の課題について述べる.

%%%%%%%%%%%%%%%%%%%%%%%%%%%%%%%%%%%%%%%%%%%%%%%%%%%%%%
\section{粗視化分子動力学(CGMD)法}
\subsection{運動方程式}
本研究におけるCGMD法では、粘土分子の単位構造とそこに水和された水分子を一つの粗視化粒子として扱う.
ここで,全部で$N$個の粗視化粒子のうち第$i$番目のものの質量を$m_i$,
速度ベクトルを$\fat{v}_i=\dot{\fat{x}}_i$,位置ベクトルを$\fat{x}_i$とすれば,
粒子$i$の運動方程式は
\begin{equation}
	m_i \dot {\fat{v}}_i =\fat{f}_i ,\ \ 
	\dot{\fat{x}}_i = \fat{v}_i, \ \ ( i =1,2,\cdots N )
	\label{eqn:eq_mot}
\end{equation}
と表される.ただし,$\fat{f}_i$は,粒子$i$に作用する力のベクトルを意味する.
また,素視化粒子の質量は,粘土分子の単位構造の質量として
\begin{equation}
	m=2.468\times 10 ^{-24}[{\rm kg}]
	\label{eqn:mass}
\end{equation}
と与えられる.$\fat{f}_i$は分子間力$\fat{f}_i^{U}$と分子内力$\fat{f}_i^{K}$の和として
\begin{equation}
	\fat{f}_i=\fat{f}_i^U+\fat{f}_i^K
	\label{eqn:f_tot}
\end{equation}
と書くことができる.粒子に作用するこれらの力の与え方は既報(2018年度共同研究報告書)
に述べた通りであり,水和水の多寡は分子間力$f_i^U$に含まれるパラメータ$\sigma$で表現される.
本年度の研究では,水和水量の増減による膨潤や膨潤圧の発生に関するシミュレーションを行う.
その方法を示すために,ここででは$\fat{f}_i^U$の詳細を述べる.その後,いかにして粘土含水系内での水分配置や,
水分の総量量を更新するかについて説明する.\\

素視化粒子間に作用する分子間力は,レナード-ジョーンズ(LJ)型のペアポテンシャル:
\begin{equation}
	U(\fat{x}_i,\fat{x}_j; \sigma) 
	= 4 \varepsilon 
	\left\{ 
	\left(\frac{\sigma}{r_{ij}}\right)^{12}
	-
	\left(\frac{\sigma}{r_{ij}}\right)^6
	\right\}, \ \ \left( r_{ij}=\left| \fat{x}_i-\fat{x}_j\right| \right)
	\label{eqn:LJ}
\end{equation}
を用いて次のように与える.
\begin{equation}
	\fat{f}_i^U(\fat{x}_i)
	=
	-\nabla_{x_i} 
	\left\{ 
		\sum_{j=1, \, j \neq i}^N U\left(\fat{x}_i,\,\fat{x}_j; \sigma \right)
	\right\}
	\label{eqn:fiU}
\end{equation}
ここに$\fat{x}_i$と$\fat{x}_j$は,現在時刻における粒子$i$と粒子$j$の位置ベクトルを,
$\varepsilon$と$\sigma$はLJポテンシャルのパラメータを,$\nabla_{x_i}=\frac{\partial}{\partial \fat{x}_i}$
はLJポテンシャルの 第一引数$\fat{x}_i$に関する勾配を表す.
LJポテンシャルの特性距離を与えるパラメータ$\sigma$は,素視化粒子間の接近限界,すなわち,粘土分子が積層
した際の層間距離を定める.従って,これを当該粒子に水和した水分の量を表す変数と考えることができる.
なお,モンモリロナイトの膨潤状態と$\sigma$の対応は,X線回折試験によって得られた粘土層間距離から
概ね表\ref{tbl:tbl_sig}のようになる.$\sigma$の値は計算途上で適宜更新され,$0.9$nm以上の値をとる.
一方,$\varepsilon$は,粗視化粒子間相互作用の強さを決定するパラメータであり、
事前に行った全原子MDシミュレーションの結果を参考に
\begin{equation}
	\varepsilon=1.0\times 10^{-19}[{\rm Nm}]
	\label{eqn:LJ_eps}
\end{equation}
と与え,こちらは定数としている.
\begin{table}[h]
	\begin{center}
	\caption{分子間相互作用ポテンシャルにおける特性距離と膨潤状態の対応.}
	\vspace{3mm}
	\begin{tabular}{c||c|c|c|c|c}
		膨潤状態 & 0層 & 1層 & 2層 & 3層 & $\cdots$\\
		\hline
		特性距離$\sigma$[{\rm nm}]& 0.9 & 1.2 & 1.5 & 1.8 & $\cdots$ \\
	\end{tabular}
	\label{tbl:tbl_sig}
	\end{center}
\end{table}
\subsection{特性距離の設定}
粘土分子表面に水和した水分の移動を許容する場合,空間的に水分量の偏りが生じる.
そのような状況を計算上で表現するためには,水和水量を与えるパラメータ$\sigma$を
粒子ペア毎に設定する必要がある.また,粘土分子の一方の面と他方の面に水和した
水分量が異なる場合,粘土分子の上下いずれのと相互作用するかを区別して
ペアポテンシャルを評価する必要がある.これらのことを考慮し,本研究では以下の
手順で$\sigma$の値を設定する.

粒子$i$と粒子$j$に関するペアポテンシャル(\ref{eqn:LJ})の特性距離$\sigma$について考える.
いま,粒子$i$が属する粘土分子の一方の面を$S^+$,他方の面を$S^-$とし,位置$\fat{x}_i(\in S^+)$
における$S^+$の法線ベクトルを$\fat{n}_i$と表す.
また,粘土分子の厚さは幅に比べて十分小さいと考え,$\fat{x}_i$における$S^-$(裏面側)の
法線ベクトルを$-\fat{n}$とする.ここで,粒子$i$の$S^+$側に水和した水和水層の厚さを
$s_i^+$,$S^-$側の厚さを$s_i^-$と書く.
このとき,粒子$j$が$\fat{n}^+$の側にある場合は$S^+$と,$-\fat{n}$の側にある場合は
$S^-$と相互作用すると考えることができる.
そこで,該当する側の面を水和水層厚$s_i$を
\begin{equation}
	s_i=\left\{
	\begin{array}{cc}
		s_i^+ & (h(\fat{x}_i,\fat{x}_j) \ge 0)\\
		s_i^- & (h(\fat{x}_j,\fat{x}_i) <0)
	\end{array}
	\right.
	\label{eqn:si_switch}
\end{equation}
\begin{equation}
	h(\fat{x}_i,\fat{x}_j) =\fat{n}_i\cdot \left(\fat{x}_j-\fat{x}_i\right)
	\label{eqn:h_sgn}
\end{equation}
で選択し,粒子$i$と粒子$j$の間のペアポテンシャルの特性距離$\sigma$を
\begin{equation}
	\sigma=s_i+s_j+\sigma_0
	\label{eqn:sig_ij}
\end{equation}
で与える.ただし,$\sigma_0(=0.9)$[nm]は無水時(0層膨潤状態)の層間距離を表す.
このようにすることで,粘土分子の表裏面を区別し,位置に応じて
異なる水和水の分布を表現することができる.
%--------------------
\begin{figure}[h]
	\begin{center}
%	\includegraphics[width=0.5\linewidth]{Figs/fig7.eps} 
	\end{center}
	\caption{
		粗視化粒子の向きを表すために用いる単位接線
		$\hat{\fat{t}}_i,\hat{\fat{t}}_j$
		および, 単位法線ベクトル.
		$\hat{\fat{n}}_i,\hat{\fat{n}}_j$. 
		$\theta_{ij}$は粒子位置$\fat{x}_i$から$\fat{x}_j$を望む方向を,
		$\theta_{ji}$は$\fat{x}_j$から$\fat{x}_i$を望む方向を,
		それぞれの粒子位置における接線ベクトルから測ったときの角度を表す.
	} 
	\label{fig:fig9}
\end{figure}
%--------------------
\subsection{時間ステッピング}
CGMD法では粒子位置$\fat{x}_i$と速度$\fat{v}_i$に加え,水和層の厚さ$s_i^\pm$が
系の時間発展に伴い更新の対象となる.以下,これらの未知量を
\begin{equation}
	\fat{V}=\left\{ \fat{v}_1,\, \fat{v}_2,\, \dots \fat{v}_N \right\}
	\label{eqn:defV}
\end{equation}
\begin{equation}
	\fat{X}=\left\{ \fat{x}_1,\, \fat{x}_2,\, \dots \fat{x}_N \right\}
	\label{eqn:defV}
\end{equation}
\begin{equation}
	\fat{\Sigma}=\left\{ \fat{s}_1^\pm,\, \fat{s}_2^\pm,\, \dots \fat{s}_N^\pm \right\}
	\label{eqn:defV}
\end{equation}
と表し,各々の時間ステッピングにともなう増分を$\Delta \fat{V},\Delta \fat{X}$,
および$\Delta \fat{\Sigma}$と書く.これらの内、位置と速度の更新は運動方程式(\ref{eqn:eq_mot})を差分法で離散化して時間積分することによって行う.
離散化のための差分スキームにはleapfrog法による中央差分を用いる.
なお,$\fat{V}$と$\fat{X}$の更新時には$\fat{\Sigma}$を一定としておく.すなわち,
\begin{equation}
	\left( \fat{V},\,\fat{X} \right)
	\rightarrow 
	\left( \fat{V},\,\fat{X} \right)
	+
	\left. \left( \Delta \fat{V},\, \Delta \fat{X} \right) \right|_{\fat{\Sigma}}
	\label{eqn:vx_update}
\end{equation}
とする. 一方,水分分布$\fat{\Sigma}$の更新にはモンテカルロ法を用い,その際$(\fat{V},\fat{X})$は一定とする.
すなわち
\begin{equation}
	\fat{\Sigma} \rightarrow \fat{\Sigma} + \left. \Delta \fat{\Sigma} \right|_{\fat{V},\fat{X}}
	\label{eqn:s_update}
\end{equation}
とする.モンテカルロ法では疑似乱数を用いて系のエネルギー$U(\fat{V},\fat{X},\fat{\Sigma})$が
低減する方向へ繰返し状態を更新する.ここでは水分の移動に関して二種類の状態更新を行う.
1つ目の状態更新では,粗視化粒子のペアを無作為に選択して,これらの粒子間で水分の授受を行うべきかを判定する.
その際,水分授受が発生したの仮定の元でエネルギーの増減$\Delta U$を計算する.
$\Delta U<0$の場合には,実際に(\ref{eqn:s_update})のように水分の状態を更新する.
この方法では系内の水分量は一定に保った状態で水分分布が変更される.
2つ目の状態更新では,一つの粗視化粒子を無作為に選択しその水分量が所定量だけ増減したと仮定する.
その仮定の元で生じるエネルギー変化$\Delta U$を計算し,$\Delta U<0$の場合に実際に当該粒子が保持する
水分量を変化させる.この状態更新では,系内に含まれる水分の総量を変化させることができる。
次節ではモンテカルロ法におけるエネルギー$U$の具体的な評価方法を述べる.
\subsection{水分量と水分分布に関するエネルギー}
水分移動に関する全エネルギーを$E$とし,これを
\begin{equation}
	E=U_{LJ} +U_{hyd} + U_n
	\label{eqn:}
\end{equation}
と表す.ここで$U_{LJ}$は粗視化粒子間相互作用の, 
$U_{hyd}$は水分子と粘土分子の水和による相互作用を,
$U_N$は系内に含まれる水分子数の増減に関するエネルギーをそれぞれ表す.
このうち$U_{LJ}$は式(\ref{eqn:LJ})を用い,
\begin{equation}
	U_{LJ}=\sum_{i\neq j} U(\fat{x}_i,\fat{x}_j;\sigma)
	\label{eqn:}
\end{equation}
で与えられ,運動方程式の積分でも考慮されているものである.
一方、$U_{hyd}$は粗視化粒子内のエネルギーで、運動方程式や粒子間相互作用
ポテンシャルには含まれていない.そこで,$U_{hyd}$は後述する方法により
別途モデル化して与える。残る$U_N$は,水分量の増減によって変化する
自由エネルギーであり、直接モデル化や計算することが難しい.
しかしながら,モンテカルロ法では状態更新に伴うエネルギーの増減だけが計算できればよいため、
必ずしも$U_N$そのものを計算できる必要はない。
そこで、$U_N$の増分$\Delta U_n$を
\begin{equation}
	\Delta U_n =\mu \Delta n
	\label{eqn:}
\end{equation}
と表す。ここに,$\mu$は外界(水分溜)の化学ポテンシャルを,
$n$は水分子の総数を表す.
なお,実際のCGMD法計算では,水分子数ではなく水和水の層厚$s=s^{+}$あるいは$s=s^{-}$を変化させる. 
そのため,$\Delta U_n$の計算にあたり,
\begin{equation}
	\Delta U_n =\tilde \mu \Delta s
	\label{eqn:}
\end{equation}
となるように,$\mu$ではなく
\begin{equation}
	\tilde \mu =\mu \frac{\Delta n}{\Delta s}
	\label{eqn:}
\end{equation}
を与えることとした.
\subsection{水和エネルギーモデル}
粒子$i$の$S^\beta$面に関する水和エネルギーを$u^\beta_i(\beta=+,-)$とする.
このとき, 系全体の水和エネルギーは
\begin{equation}
	U_{hyd}=\sum_{i=1}^N\sum_{\beta=+,-} u_i^{\beta}
	\label{eqn:}
\end{equation}
で与えられる.いま,$u^+_i$と$u^-_i$を与える関数は粒子や面の向きに依らないと考えてよく,
\begin{equation}
	u^\beta_i= u(s^\beta), \ \ (s^\beta>0,\, \beta=+,-)
	\label{eqn:}
\end{equation}
としてよい.そこで,$u$を水和水層の厚さ$s$の関数としてモデル化する.
$u(s)$は$s$に対して大局的には減少する有界な関数である.
また$u(0)=0$としてよいため,ここでは次の指数関数を用いて与える.
\begin{equation}
	u(s)=u_{\infty} \left(1-e^{-\gamma s} \right)
	\label{eqn:u_s}
\end{equation}
ここに$u_{\infty}$は$s\rightarrow \infty$(無限膨潤)での水和エネルギーを表す.
$\gamma$は
\begin{equation}
	u'(s)=\frac{du}{ds}=\gamma u_{\infty} e^{-\gamma s}
	\label{eqn:ud_s}
\end{equation}
より,水和エネルギーの$s$に対する変化率を決めるパラメータである。
ここでは,$u'(s)$の半減距離$s=s_b$を与え,
\begin{equation}
	\frac{u'\left(s_b\right)}{u'(0)}=e^{-\gamma s_b}=\frac{1}{2} 
	\ \ \Rightarrow \ \
	\gamma=\frac{\log 2}{s_b}
	\label{eqn:}
\end{equation}
から$\gamma$の値を決定した.
$u(s)$と$u'(s)$は図\ref{fig:fig0}に示すような形状をしている.
このグラフでは横軸を半減距離$s_b$で無次元化して表示しているため,
$u(s),u'(s)$とも$s/s_b=1$において$u(0),u'(0)$から値が半減している
様子が示されている.
%--------------------
\begin{figure}[h]
	\begin{center}
	\includegraphics[width=1.0\linewidth]{Figs/fig0.eps} 
	\end{center}
	\caption{
		水和エネルギー$u$とその導関数$u'$.
		$s$は水和水層の厚さ,$s_b$は$-u'$の半減距離,
		$\gamma$は減衰率($e-\gamma s_b=1/2$)を表す.
	} 
	\label{fig:fig0}
\end{figure}
%--------------------

%--------------------
\section{シミュレーションモデル}
以下では,化学ポテンシャル$\tilde \mu$を一定としたときの,粘土含水系への
水分の出入りや膨潤圧の発生挙動を調べることを目的としたCGMDシミュレーション
の例を示す.シミュレーションは,次の3つのステップで行う.
\begin{enumerate}
\item
	ランダムに配置した粘土分子の圧縮凝集
\item
	凝集した粘土含水系の冷却
\item
	温度,体積,化学ポテンシャル一定での緩和
\end{enumerate}
これら3つのステップにおけるシミュレーションの詳細と結果を順に示す.
\subsection{水和粘土分子の圧縮凝集}
CGMDシミュレーションの初期モデルを図\ref{fig:fig1}に示す.
この図は,時刻$t=0$における粘土分子の配置と粘土分子幅(粒径)のヒストグラムを示したものである.
周期構造を仮定し,粘土分子の初期配置を示した図(a)には,そのユニットセルが破線で示されている.
ユニットセルに含まれる粘土分子の位置は一様乱数で与え,粒径はガウス分布を使って与えたもので,
分子数,粗視化粒子数,粒径の平均値と標準偏差は以下のようである.
\begin{itemize}
	\item 粘土分子数: 80
	\item 粗視化粒子数: 3,194
	\item 平均粒径 (標準偏差): 40 (10)[{\rm nm}]
\end{itemize}
初期状態では,ユニットセルのサイズは200$\times$200[nm$^2$]の正方形で,
これを1[ns]の間に各辺を65\%等方的に圧縮する.その結果,最終的には70$\times$70[nm$^2$]の
正方形セルに,80の粘土分子が充填された組織構造が得られる.
なお,全ての粘土分子は初期状態で二層膨潤に相当する水和水を持つものとしている.
これは粗視化粒子に設定する水和水層の厚さを,一律に$\sigma_i^\pm=1.5$[nm]とすることを意味する.
ユニットセルの圧縮過程においては,全ポテンシャルエネルギー$U_{LJ}$が低下する場合には
近接する粗視化粒子間で水和水の交換がおきることを許容した計算を行う.
そのため,粘土含水系が有する全水分量は一定に保たれるものの,最終的に得られた
凝集構造において水分分布は一様でなく,二層膨潤の状態が維持される保証はない.
ただし,水和水層の厚さは$s_i^\pm$は$0.45$[nm],すなわち3層膨潤を上限としている.
これは,極端に多量の水分を持つ粒子が発生することを防ぐための便宜的な措置である.
%
なお,圧縮時に系の温度は制御していない.そのため,粗視化粒子はユニットセルの圧縮の過程で
運動エネルギーを得て系の温度が次第に上昇する.また,ユニットセルの圧縮を終了する
時刻$t=1$[ns]では系は平衡状態には至っていないことに注意する.
%--------------------
\begin{figure}[h]
	\begin{center}
	\includegraphics[width=0.8\linewidth]{Figs/fig1.eps} 
	\end{center}
	\caption{
		解析モデル. (a)粘土分子の初期分布.(b)粘土分子幅(粒径)のヒストグラム. 
	} 
	\label{fig:fig1}
\end{figure}
%--------------------

以上の条件で行った計算の結果を図\ref{fig:fig2}に示す.
この図は粘土分子の瞬間構造(スナップショット)を0.2ns毎に示したもので,
ユニットセルの圧縮による粘土分子の凝集挙動をみるためのものである.
この結果に示されるように,当初はほぼ均等に分散していた粘土分子が,
ユニットセルの圧縮にともない互いに次第に接近する.
接近した分子どうしが分子間力で相互作用しはじめると,粘土分子は
不均一な力を受け,屈曲振動をおこしながら少しずつ積層する.
分子の積層が進むにつれて大きな間隙が形成されるが,それらの間隙
も圧縮が進行するにつれて収縮する.同時に,積層した粘土分子群の
屈曲が大きくなり,最終的には完全に折りたたまれた状態で充填される
分子も現れる.積層した粘土分子間の細長い空隙に比べて相対的に大きな空隙は,このような折りたたまれた分子が囲い込む
領域として形成されたものである.
\begin{figure}[h]
	\begin{center}
	\includegraphics[width=1.0\linewidth]{Figs/fig2.eps} 
	\end{center}
	\caption{
		ユニットセルの圧縮に伴う粘土分子の凝集挙動.
	} 
	\label{fig:fig2}
\end{figure}
\subsection{凝集した粘土含水系の冷却}
前述したように,図\ref{fig:fig2}-(f)に示した状態は平衡状態にはなく,
急激な圧縮によりエネルギーも高い状態にある.そこで,粗視化粒子の
運動エネルギーを時間に関して一定の割合で減少させることで次式
で与えられる系の温度$T$を下げる.
\begin{equation}
	T=\frac{K}{\frac{3}{2}nk_B}
	\label{eqn:Temp}
\end{equation}
ここに,$n$は粒子数,$K$は運動エネルギー,$k_B$はボルツマン定数を表す.
その後,定温,定積条件で一定時間系を緩和させる.
ここでは,圧縮凝集直後の状態(図\ref{fig:fig2}-(f)からスタートして
250[ps]の間に300[K]まで冷却する,次に,750[ps]間, 温度を300[K]に
保って緩和を行う.図\ref{fig:fig8}は,このときのエネルギーの推移を
示したもので,ポテンシャルエネルギー$U_{LJ}$と運動エネルギー,
両者の和である全エネルギーを示している.冷却の開始時点,すなわち
圧縮凝集終了直後は運動エネルギー, ポテンシャルエネルギーともに
高い状態にあるが,冷却と続く緩和により,いずれのエネルギーもほぼ
単調に減少していることが分かる.特に運動エネルギーに関しては,
期待した通り,250[ps]間にほとんど直線的に変化し,その後は一定の値を
保っていることが分かる.一方,ポテンシャルエネルギーは冷却終了後も
非常にゆっくりと減少を続けている.これはより低いエネルギーとなる
水分配置の探索が継続し,緩和に非常に長い時間がかかることを示している.
%--------------------
\begin{figure}[h]
	\begin{center}
	\includegraphics[width=0.7\linewidth]{Figs/fig8.eps} 
	\end{center}
	\caption{
		圧縮凝集後の冷却によるエネルギーの変化.
	} 
	\label{fig:fig8}
\end{figure}
%--------------------
\begin{figure}[h]
	\begin{center}
	\includegraphics[width=1.0\linewidth]{Figs/fig3.eps} 
	\end{center}
	\caption{
		圧縮凝集後の冷却による組織構造の変化.
	} 
	\label{fig:fig3}
\end{figure}
%--------------------

図\ref{fig:fig3}に冷却と緩和にともなう組織構造の変化を示す.
この図には,冷却開始時($t=0$)と冷却終了後($t=400$[ps]),
定温定積での緩和を終了した時点$(t=1000$[ps])での粘土分子配置
を示している.粘土分子は,当初より非常に密に充填されているため,
冷却や緩和によって粘土分子の全体的な配置は大きく変化していない.
特に図\ref{fig:fig3}-(b)と(c)に示した結果は互いにほとんど差がない.
一方(a)と(b)の結果を比較すると,主として周囲と比べて相対的に大きな
空隙が残された箇所で変化があることが分かる.
(a)の図で(i),(ii)と示したのはそのような領域の例で,冷却と緩和により
一部の空洞は消失し,消失せずに残る空洞も位置や形状が若干変化している
ことが分かる.図\ref{fig:fig8}に示した冷却中(250[ps]まで)の
ポテンシャルエネルギーの変化は,このような組織構造の変化によって
生じたものと理解できる.

\section{水分量変化を許容したCGMDシミュレーション}
最後に,定温定積かつ化学ポテンシャル一定の元で行った,CGMDシミュレーションの結果を示す.
ユニットセルのサイズは初期モデルの状態で固定し,温度を300[K]に保ったまま
一定の化学ポテンシャル$\tilde \mu$のもとで1[ns]の間,水分の出入りを
許す状態で粘土含水系を緩和させる.

図\ref{fig:fig4}は,$\tilde \mu=0$の場合について組織構造の変化を示したものである.
この場合,水分量の増減によるエネルギーの変化はなく,水和エネルギーの減少が
ポテンシャルエネルギーの増加を上回る限り系内部の水分量は増加する.
その結果、積層した粘土層間の距離が大きくなり,かつ層間距離が均一化する様子が見られる.
例えば,図\ref{fig4}-(a)において(i)と(ii)の破線で囲んだ箇所は,当初,層間距離が
にばらつきが見られるが,これらの箇所吸水により最終的には層間距離がほぼ一定の値に
揃っている.また,粘土分子が折りたたまれて出来た空隙部分も水分の増加に伴い,その
大きさは次第に小さくなっている.
%
これに対して図\ref{fig:fig5}は,化学ポテンシャルを
\begin{equation}
	\frac{\tilde \mu}{u'(0)}=
	\frac{\tilde \mu}{\gamma u_\infty}=\frac{1}{5}
	\label{eqn:mu02}
\end{equation}
とした場合の結果を示したものである.式(\ref{eqn:mu02})は,
水和エネルギー$u(s)$の勾配の最大値$u'(0)=\gamma u_{\infty}$を基準にして
化学ポテンシャルを与えることを意図したものである.
ここでは,水分量の変動によるエネルギーの増減が,
水和エネルギーの増減を凌駕して完全な排水が起こることが無い程度の値に
化学ポテンシャルの大きさを設定している.
実際,図\ref{fig:fig5}の結果では,一部の水分が排出されることで
空隙が大きくなる方向へ組織構造が変化しているが,水分量の変化は
あまり大きくないため,粘土分子の相対的な配置や屈曲の変化は劇的なものではない.
%図$\tilde \mu=0$の場合よりも顕著な組織構造の変化がおきているが,
%水分量の変化粘土分子の全体的な分布には大きな変化が置きていない.
ただし,詳しくみれば,水分量の減少によって生まれたスペースを使って粘土分子が
若干移動することで,屈曲のが解消される箇所や空洞が統合あるいは消失する様子が
示されている.
%--------------------
\begin{figure}[h]
	\begin{center}
	\includegraphics[width=1.0\linewidth]{Figs/fig4.eps} 
	\end{center}
	\caption{
		定積,定温,化学ポテンシャル一定での緩和挙動($\tilde \mu =0$の場合).
	} 
	\label{fig:fig4}
\end{figure}
%--------------------
%--------------------
\begin{figure}[h]
	\begin{center}
	\includegraphics[width=1.0\linewidth]{Figs/fig5.eps} 
	\end{center}
	\caption{
		定積,定温,化学ポテンシャル一定での緩和挙動($\tilde \mu =\frac{1}{5}\gamma u_{\infty}$の場合).
	} 
	\label{fig:fig5}
\end{figure}
%--------------------
次に,各粗視化粒子が有する水分量を図\ref{fig:fig6}に示す.
この図は,横軸に粗視化粒子番号を,縦軸に水和水層の厚さ$s_i^+$と$s^i_-$を示した
ものである.一つの分子を構成する粗視化粒子には連続する粒子番号が振られており,
この図では,各粒子が属する分子が変化するたびに異なる色の実線で水和水層の厚さを
プロットしている.なお$s^+_i$は実線で,$s^-_i$は破線で表示しているが,
両者はほとんどの場合近い値となっておりグラフ上ではほとんど区別がつかなくなっている.
(a),(b),(c)はそれぞれ,化学ポテンシャル一定での緩和の開始時,$\tilde \mu =0$と
$\tilde \mu =\frac{1}{5}\gamma u_{\infty}$での緩和を終了させた時点での結果を示している.
図\ref{fig:fig7}はこれらの結果を$s^\pm_i$のヒストグラムとして表示したものである。
これらの図から明らかな通り,当初水分量が大きく偏った状態にあるものの,
緩和後には水分が粒子に概ね均等に配分されている.特に,化学ポテンシャルの値が相対的
に大きな図\ref{fig:fig6},\ref{fig:fig7}-(c)のケースでは,全ての粒子で
一定値$s^{^pm}_i\simeq =0.235$[nm]に近づいており、これは1.5層膨潤程度に相当する.
なお,$\tilde \mu=0$の場合には,いくつかの粘土分子で平均値よりも明らかに大きな
$s^\pm_i$の値を取るものがある.これは,緩和開始時の水分分布の偏りが残留したもので,
粘土分子影響を受けたもので,水分の増加にともない,粘土分子が移動する余地がほとんど無く
層間距離をほとんど変化させることができない箇所に水分が充填された結果と考えることができる.
%--------------------
\begin{figure}[h]
	\begin{center}
	\includegraphics[width=0.8\linewidth]{Figs/fig6.eps} 
	\end{center}
	\caption{
		各粗視化粒子における水和水層の厚さ$s_i^+,s_i^-$.
		(a)緩和開始時,(b)$\tilde \mu=0$, (c)$\tilde \mu =\frac{1}{5}\gamma u_\infty.$
	} 
	\label{fig:fig6}
\end{figure}
%--------------------
%--------------------
\begin{figure}[h]
	\begin{center}
	\includegraphics[width=0.6\linewidth]{Figs/fig7.eps} 
	\end{center}
	\caption{
		各粗視化粒子における水和水層の厚さ$s_i^+,s_i^-$のヒストグラム.
		(a)緩和開始時,(b)$\tilde \mu=0$, (c)$\tilde \mu =\frac{1}{5}\gamma u_\infty.$
	} 
	\label{fig:fig7}
\end{figure}
%--------------------
ここでは体積が一定の条件で緩和を行っている。
そのため、緩和の過程で吸水が起きる場合には膨潤圧が,排水が起きる場合は
減圧が生じる.
図\ref{fig:fig9}と図\ref{fig:fig10}はこのことを示したものである。
これらの図には、温度と応力,エネルギーの推移が示されている。
温度は300[K]に保たれているが,水分の移動により粗視化粒子の運動状況が変化する
ため、300[K]前後で10度程度のゆらぎが見られる。
応力は,せん断力成分$\sigma_{12}$は化学ポテンシャルの値によらず
一貫してほぼ零だが,直応力成分$\sigma_{11}$と$\sigma_{22}$は$\tilde \mu=0$では
単調に増加$\tilde \mu =1/2\gamma u_\infty$では減少した後、若干のオーバーシュート
を経て一定値に至る様子が現れている.
この間、エネルギーは,ポテン車ry
%--------------------
\begin{figure}[h]
	\begin{center}
	\includegraphics[width=1.0\linewidth]{Figs/fig9.eps} 
	\end{center}
	\caption{
		化学ポテンシャル一定の元での緩和に伴う
		(a)温度, (b)応力と(c)エネルギーの変化 ($\tilde \mu=0$).
	} 
	\label{fig:fig9}
\end{figure}
%--------------------
%--------------------
\begin{figure}[h]
	\begin{center}
	\includegraphics[width=1.0\linewidth]{Figs/fig10.eps} 
	\end{center}
	\caption{
		化学ポテンシャル一定の元での緩和に伴う(a)温度, (b)応力と (c) エネルギーの変化
		($\tilde \mu =\frac{1}{5}\gamma u_\infty).$
	} 
	\label{fig:fig10}
\end{figure}
%--------------------
%--------------------
\begin{figure}[h]
	\begin{center}
	\includegraphics[width=0.8\linewidth]{Figs/fig11.eps} 
	\end{center}
	\caption{
		caption.
	} 
	\label{fig:fig11}
\end{figure}
%--------------------
%--------------------
\begin{figure}[h]
	\begin{center}
	\includegraphics[width=1.0\linewidth]{Figs/fig12.eps} 
	\end{center}
	\caption{
		caption.
	} 
	\label{fig:fig12}
\end{figure}
%--------------------


\section{まとめと今後の課題}
本研究では,粘土含水系の組織形成シミュレーションにおいて水分量を変化させることができるように,
水和エネルギーと水分量の関係をモデル化し粗視化MD(CGMD)法への組み込みを行った.
ここで拡張を行ったCGMD法では,粗視化粒子の運動方程式を積分する途上で水分の移動を
水和エネルギーを考慮したモンテカルロ法によって行う.その際,粘土分子への水和エネルギーと
水分量に比例したエネルギーを評価して状態遷移を繰り返すことで,より安定な水分量と水分配置を探索する.
このような方法で定積,定温,化学ポテンシャル一定の条件において粘土含水系の緩和シミュレーション
を行ったところ,化学ポテンシャルが小さいときには吸水による膨潤圧が発生し,
化学ポテンシャルが大きいときには脱水により圧力の低下と組織構造の有意な変化が生じることが確認された.
以上の結果は,本手法を用いて所定の温度や体積,化学ポテンシャルの下で粘土が保持できる
水分量が求められることを意味する.
すなわち,環境の変化によってどのように吸水,脱水が生じるかを調べることが可能とでき,
このことから,本研究の手法は粘土の膨潤や不飽和浸透問題の解析にも有用なものとなることが期待される.
そのためには,水和エネルギーの詳細なモデル化を行うことが今後の課題となる.
特に,ナトリウム型モンモリロナイトは相対湿度に対して階段状の膨潤曲線を
示すことから,これを再現するような水和エネルギーモデルの開発が緩衝材の
挙動を調べるためには必要となる.また,分子間相互作用に関するポテンシャルエネルギーと
水和エネルギーの比率を正確に評価することや,CGMD法における温度と物理的な温度の対応をつける
ことも,今後膨潤圧や弾性係数を実測値と比較する際に必要となる.
%
%\begin{thebibliography}{99}
%\bibitem{Warren}
%	B.E. Warren, "X-ray diffraction", Dover Publication, New York, 1990.
%\bibitem{Koide}
%	小出昭一郎,"物理現象のフーリエ解析", ちくま学芸文庫, 2018.
%\end{thebibliography}
\end{document}
%%%%%%%%%%%%%%%%%%%%%%%%%%%%%%%%%%%%%%%%%%%%%%%%%%%%%%

