%--------------------
\section{シミュレーションモデル}
以下では,化学ポテンシャル$\tilde \mu$を一定としたときの,粘土含水系への
水分の出入りや膨潤圧の発生挙動を調べることを目的としたCGMDシミュレーション
の例を示す.シミュレーションは,次の3つのステップで行う.
\begin{enumerate}
\item
	ランダムに配置した粘土分子の圧縮凝集
\item
	凝集した粘土含水系の冷却
\item
	温度,体積,化学ポテンシャル一定での緩和
\end{enumerate}
これら3つのステップにおけるシミュレーションの詳細と結果を順に示す.
\subsection{水和粘土分子の圧縮凝集}
CGMDシミュレーションの初期モデルを図\ref{fig:fig1}に示す.
この図は,時刻$t=0$における粘土分子の配置と粘土分子幅(粒径)のヒストグラムを示したものである.
周期構造を仮定し,粘土分子の初期配置を示した図(a)には,そのユニットセルが破線で示されている.
ユニットセルに含まれる粘土分子の位置は一様乱数で与え,粒径はガウス分布を使って与えたもので,
分子数,粗視化粒子数,粒径の平均値と標準偏差は以下のようである.
\begin{itemize}
	\item 粘土分子数: 80
	\item 粗視化粒子数: 3,194
	\item 平均粒径 (標準偏差): 40 (10)[{\rm nm}]
\end{itemize}
初期状態では,ユニットセルのサイズは200$\times$200[nm$^2$]の正方形で,
これを1[ns]の間に各辺を65\%等方的に圧縮する.その結果,最終的には70$\times$70[nm$^2$]の
正方形セルに,80の粘土分子が充填された組織構造が得られる.
なお,全ての粘土分子は初期状態で二層膨潤に相当する水和水を持つものとしている.
これは粗視化粒子に設定する水和水層の厚さを,一律に$\sigma_i^\pm=1.5$[nm]とすることを意味する.
ユニットセルの圧縮過程においては,全ポテンシャルエネルギー$U_{LJ}$が低下する場合には
近接する粗視化粒子間で水和水の交換がおきることを許容した計算を行う.
そのため,粘土含水系が有する全水分量は一定に保たれるものの,最終的に得られた
凝集構造において水分分布は一様でなく,二層膨潤の状態が維持される保証はない.
ただし,水和水層の厚さは$s_i^\pm$は$0.45$[nm],すなわち3層膨潤を上限としている.
これは,極端に多量の水分を持つ粒子が発生することを防ぐための便宜的な措置である.
%
なお,圧縮時に系の温度は制御していない.そのため,粗視化粒子はユニットセルの圧縮の過程で
運動エネルギーを得て系の温度が次第に上昇する.また,ユニットセルの圧縮を終了する
時刻$t=1$[ns]では系は平衡状態には至っていないことに注意する.
%--------------------
\begin{figure}[h]
	\begin{center}
	\includegraphics[width=0.8\linewidth]{Figs/fig1.eps} 
	\end{center}
	\caption{
		解析モデル. (a)粘土分子の初期分布.(b)粘土分子幅(粒径)のヒストグラム. 
	} 
	\label{fig:fig1}
\end{figure}
%--------------------

以上の条件で行った計算の結果を図\ref{fig:fig2}に示す.
この図は粘土分子の瞬間構造(スナップショット)を0.2ns毎に示したもので,
ユニットセルの圧縮による粘土分子の凝集挙動をみるためのものである.
この結果に示されるように,当初はほぼ均等に分散していた粘土分子が,
ユニットセルの圧縮にともない互いに次第に接近する.
接近した分子どうしが分子間力で相互作用しはじめると,粘土分子は
不均一な力を受け,屈曲振動をおこしながら少しずつ積層する.
分子の積層が進むにつれて大きな間隙が形成されるが,それらの間隙
も圧縮が進行するにつれて収縮する.同時に,積層した粘土分子群の
屈曲が大きくなり,最終的には完全に折りたたまれた状態で充填される
分子も現れる.積層した粘土分子間の細長い空隙に比べて相対的に大きな空隙は,このような折りたたまれた分子が囲い込む
領域として形成されたものである.
\begin{figure}[h]
	\begin{center}
	\includegraphics[width=1.0\linewidth]{Figs/fig2.eps} 
	\end{center}
	\caption{
		ユニットセルの圧縮に伴う粘土分子の凝集挙動.
	} 
	\label{fig:fig2}
\end{figure}
\subsection{凝集した粘土含水系の冷却}
前述したように,図\ref{fig:fig2}-(f)に示した状態は平衡状態にはなく,
急激な圧縮によりエネルギーも高い状態にある.そこで,粗視化粒子の
運動エネルギーを時間に関して一定の割合で減少させることで次式
で与えられる系の温度$T$を下げる.
\begin{equation}
	T=\frac{K}{\frac{3}{2}nk_B}
	\label{eqn:Temp}
\end{equation}
ここに,$n$は粒子数,$K$は運動エネルギー,$k_B$はボルツマン定数を表す.
その後,定温,定積条件で一定時間系を緩和させる.
ここでは,圧縮凝集直後の状態(図\ref{fig:fig2}-(f)からスタートして
250[ps]の間に300[K]まで冷却する,次に,750[ps]間, 温度を300[K]に
保って緩和を行う.図\ref{fig:fig8}は,このときのエネルギーの推移を
示したもので,ポテンシャルエネルギー$U_{LJ}$と運動エネルギー,
両者の和である全エネルギーを示している.冷却の開始時点,すなわち
圧縮凝集終了直後は運動エネルギー, ポテンシャルエネルギーともに
高い状態にあるが,冷却と続く緩和により,いずれのエネルギーもほぼ
単調に減少していることが分かる.特に運動エネルギーに関しては,
期待した通り,250[ps]間にほとんど直線的に変化し,その後は一定の値を
保っていることが分かる.一方,ポテンシャルエネルギーは冷却終了後も
非常にゆっくりと減少を続けている.これはより低いエネルギーとなる
水分配置の探索が継続し,緩和に非常に長い時間がかかることを示している.
%--------------------
\begin{figure}[h]
	\begin{center}
	\includegraphics[width=0.7\linewidth]{Figs/fig8.eps} 
	\end{center}
	\caption{
		圧縮凝集後の冷却によるエネルギーの変化.
	} 
	\label{fig:fig8}
\end{figure}
%--------------------
\begin{figure}[h]
	\begin{center}
	\includegraphics[width=1.0\linewidth]{Figs/fig3.eps} 
	\end{center}
	\caption{
		圧縮凝集後の冷却による組織構造の変化.
	} 
	\label{fig:fig3}
\end{figure}
%--------------------

図\ref{fig:fig3}に冷却と緩和にともなう組織構造の変化を示す.
この図には,冷却開始時($t=0$)と冷却終了後($t=400$[ps]),
定温定積での緩和を終了した時点$(t=1000$[ps])での粘土分子配置
を示している.粘土分子は,当初より非常に密に充填されているため,
冷却や緩和によって粘土分子の全体的な配置は大きく変化していない.
特に図\ref{fig:fig3}-(b)と(c)に示した結果は互いにほとんど差がない.
一方(a)と(b)の結果を比較すると,主として周囲と比べて相対的に大きな
空隙が残された箇所で変化があることが分かる.
(a)の図で(i),(ii)と示したのはそのような領域の例で,冷却と緩和により
一部の空洞は消失し,消失せずに残る空洞も位置や形状が若干変化している
ことが分かる.図\ref{fig:fig8}に示した冷却中(250[ps]まで)の
ポテンシャルエネルギーの変化は,このような組織構造の変化によって
生じたものと理解できる.
