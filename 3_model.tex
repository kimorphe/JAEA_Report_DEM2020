%--------------------
\section{シミュレーションモデル}
以下では,化学ポテンシャル$\tilde \mu$を一定の条件で粘土含水系への
水分の出入りや膨潤圧の発生挙動をみるためのCGMDシミュレーションを行う.
CGMDシミュレーションは,次の手順で行う.
\begin{enumerate}
\item
	ランダムに配置した粘土分子の圧縮凝集
\item
	所定温度への凝集系の冷却
\item
	温度,体積,化学ポテンシャル一定での緩和
\end{enumerate}
これら3段階の過程を経て行ったシミュレーションの結果を以下順に示す.
\subsection{水和粘土分子の圧縮凝集}
CGMDシミュレーションの初期モデルを図\ref{fig:fig1}に示す.
この図は,粘土分子の初期(時刻$t=0$)配置と,粘土分子幅(粒径)のヒストグラムを示したものである.
いずれのモデルも周期構造が仮定されており,粘土分子の初期配置を示した図には,
周期構造のユニットセルが破線で示されている.
ユニットセルに含まれる粘土分子の位置は一様乱数で,分子の長さはガウス分布で与えた.
分子数や分子長さ(粒径)のその平均値等は以下のように設定している.
\begin{itemize}
	\item 粘土分子数: 80
	\item 粗視化粒子数: 3,194
	\item 平均粒径 (標準偏差): 40 (10)[{\rm nm}]
\end{itemize}
初期状態でのユニットセルのサイズは
200$\times$200[nm]の正方形とし,1[ns]の間にユニットセルを等方的に65\%圧縮する.
その結果,最終的には70$\times$70[nm]の正方形ユニットセルに約80の粘土分子が
充填された組織構造が得られる.なお,全ての粘土分子は初期状態で二層膨潤に相当する水和水を
持つものとしている.
これは粗視化粒子に設定する水和水層の厚さを,一律に$\sigma_i^\pm=1.5$[nm]とすることを意味する.
ただし,ユニットセルの圧縮過程において,近接する粗視化粒子間では,
モデル全体のポテンシャルエネルギーが下がる場合方向へは,水和水の交換がおきることを
許容したモデルとなっている.そのため,全水分量は一定に保たれるものの,最終的に得られた
凝集構造において水分分布は一様でなく,二層膨潤の状態が維持される保証はない.
なお,圧縮時に系の温度は制御していない.そのため,圧縮によって粗視化粒子は運動エネルギーを得て
系の温度は次第に上昇する.また,圧縮終了時刻$t=1$[ns]では系は平衡状態には至っていない.
%--------------------
\begin{figure}[h]
	\begin{center}
	\includegraphics[width=0.8\linewidth]{Figs/fig1.eps} 
	\end{center}
	\caption{
		解析モデル. (a)粘土分子の初期分布.(b)粘土分子幅のヒストグラム. 
	} 
	\label{fig:fig1}
\end{figure}
%--------------------

以上の条件で行った計算の結果を図\ref{fig:fig2}に示す.
この図は粘土分子分布のスナップショットを0.2ns毎に示したもので,ユニットセルの圧縮に
よる粘土分子の凝集挙動を示している.
当初ほぼ均等に分散していた粘土分子は,ユニットセルの圧縮により互いに
次第に接近する.接近した分子どうしが分子間力で相互作用しはじめると,
粘土分子の不均一な力を受け,屈曲振動しながら少しずつ積層する.
積層した分子の間には明らかな間隙が形成されるが,圧縮によりそれらの間隙
は収縮し,それに伴い、積層した粘土分子の屈曲が大きくなり,最終的には
完全に折りたたまれた状態でユニットセル内に充填される分子が現れ,
相対的な大きな空隙はこのような折りたたまれた分子の箇所に残される様子が見られる.
\begin{figure}[h]
	\begin{center}
	\includegraphics[width=1.0\linewidth]{Figs/fig2.eps} 
	\end{center}
	\caption{
		ユニットセルの圧縮に伴う粘土分子の凝集挙動.
	} 
	\label{fig:fig2}
\end{figure}
\subsection{凝集した粘土含水系の冷却}
図\ref{fig:fig2}の(f)に示した状態は平衡状態になく,急激な圧縮によりエネルギー
が高い状態にある.そこで,粗視化粒子の運動エネルギーを少しずつ減少させる
ことで冷却し,その後一定の温度に保持して系を定温定積で緩和させる.
ここでは,圧縮凝集直後の状態から,250[ps]の間に300[K]まで運動エネルギーを
逓減して冷却し,その後750[ps]の間300[K]の状態を保って緩和を行った.
図\ref{fig:fig8}は,このときのエネルギーの推移を示したもので,
ポテンシャルエネルギー,運動エネルギーと両者の和である全エネルギーを
示している.冷却開始の段階,すなわち凝集直後は運動エネルギー,
ポテンシャルエネルギーともに高い状態にあるが,緩和と冷却により
ほぼ単調に減少していくことが分かる.運動エネルギーに関しては,
期待した通り250[ps]の間にほとんど直線的に変化し,その後は一定の値を
保っていることが分かる.一方、ポテンシャルエネルギーは冷却終了後も
非常にゆっくりと減少を続けている,これはより低いエネルギーとなる
水分配置の探索が継続していることを示している.
%--------------------
\begin{figure}[h]
	\begin{center}
	\includegraphics[width=0.7\linewidth]{Figs/fig8.eps} 
	\end{center}
	\caption{
		圧縮凝集後の冷却と緩和によるエネルギーの変化.
	} 
	\label{fig:fig8}
\end{figure}
%--------------------
\begin{figure}[h]
	\begin{center}
	\includegraphics[width=1.0\linewidth]{Figs/fig3.eps} 
	\end{center}
	\caption{
		圧縮凝集後の冷却と緩和による組織構造の変化.
	} 
	\label{fig:fig3}
\end{figure}
%--------------------
図\ref{fig:fig3}は,この間の組織構造の変化を示したもので,
励起役開始時と途上,終了時点における粘土分子の配置を表している.
粘土分子は非常に密に充填されているため,冷却-緩和によって
全体的な配置はほとんど変化していない.
特に(b)と(c)に示した結果は互いにほとんど差がない.
一方(a)と(b)を比較すると,主として周囲と比べて相対的に大きな空隙が
残された箇所で変化があることが分かる.
(a)の図で(i),(ii)と示したのはそのような領域の一つで,
冷却と緩和により,一部の空洞は消失し,残った空洞も位置や形状が
若干変化していることが分かる.
図\ref{fig:fig8}に示したポテンシャルエネルギーの変化は,
このような組織構造の変化によって生じたものと理解できる.
