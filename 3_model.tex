%--------------------
\subsection{粘土含水系モデルとメソMDシミューレションの条件}
本研究で組織構造解析に用いる,2種類の粘土含水系メソMDモデル(モデル1と2)を
図\ref{fig:fig1}と図\ref{fig:fig2}示す.
これらの図は,粘土分子の初期状態(時刻$t=0$)での分布と,粘土分子幅(粒径)の
ヒストグラムを示したものである.
いずれのモデルも周期構造が仮定されており,粘土分子の初期配置を示した図には,
周期構造のユニットセルが破線で示されている.
2つのモデルの主たる違いは粒径分布にあり,各々のモデルにおける粘土分子数と
その粒径は,正規分布で以下のように設定している.
\begin{itemize}
	\item 粘土分子数: 80
	\item 粗視化粒子数: 3,194
	\item 平均粒径 (標準偏差): 40 (10)[{\rm nm}]
\end{itemize}
これらの数値から明らかなように,モデル2はモデル1に比べて粒径分布の幅が,標準偏差で
3倍弱広く設定されている.いずれのモデルでも,初期状態でのユニットセルのサイズは
200$\times$200[nm]の正方形とし,1[ns]の間にユニットセルを等方的に65\%圧縮する.
その結果,最終的には70$\times$70[nm]の正方形ユニットセルに約80の粘土分子が
充填された組織構造が得られる.なお,全ての粘土分子は初期状態で二層膨潤に相当する水和水を
持つものとしている.ただし,ユニットセルの圧縮過程において,近接する粗視化粒子間では,
モデル全体のポテンシャルエネルギーが下がる場合方向へは,水和水が移動することを
許容したモデルとなっている.そのため,最終的に得られた組織構造において水分分布は一様でなく,
必ずしも平均的に二層膨潤の状態が維持される保証はない.なお,圧縮は等温条件で進められ,
系の温度は,粗視化粒子の運動エネルギーを各時間ステップでスケーリングすることで一定値を
保つように制御した.
%--------------------
\begin{figure}[h]
	\begin{center}
	\includegraphics[width=0.8\linewidth]{Figs/fig1.eps} 
	\end{center}
	\caption{
		解析モデル. (a)粘土分子の初期分布.(b)粘土分子幅のヒストグラム. 
	} 
	\label{fig:fig1}
\end{figure}
%--------------------
\subsection{圧縮凝集過程のシミュレーション結果}
図\ref{fig:fig3}と図\ref{fig:fig4}に,上記の計算条件で行ったメソMDの解析結果を示す.
これらの図は,粘土分子分布のスナップショットを0.2ns毎に示したもので,ユニットセルの等温圧縮に
よる粘土分子の凝集挙動を示している.いずれのモデルでも,圧縮過程の比較的早い段階において
近接する粘土分子が積層構造を作るが,積総する分子の数があまり大きくならないことが分かる.
また,積層した分子外に残された大きな間隙が,ユニットセルの圧縮に伴い次第に縮小して最終的には
ほとんど消失する様子が見られる.粘土分子の配置は,これら2つのモデルの各時刻で当然互いに異なる.
しかしながら,凝集挙動や最終的にえられた組織構造に本質的な差があるかどうかは一見しただけでは
明らかでない.次節では,このようにして得られた組織構造を処理し,X線回折パターン等の形で比較する
ことで,2つの組織構造モデルには有意な差が見られることを明らかにする.
\begin{figure}[h]
	\begin{center}
	\includegraphics[width=1.0\linewidth]{Figs/fig2.eps} 
	\end{center}
	\caption{
		ユニットセルの圧縮に伴う粘土分子の凝集挙動.
	} 
	\label{fig:fig2}
\end{figure}
\begin{figure}[h]
	\begin{center}
	\includegraphics[width=1.0\linewidth]{Figs/fig3.eps} 
	\end{center}
	\caption{
		圧縮凝集終了後の緩和挙動.
	} 
	\label{fig:fig3}
\end{figure}
%--------------------
