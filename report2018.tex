\documentclass[11pt,a4j]{jarticle}
\usepackage[dvipdfmx]{graphicx,color}
\usepackage{wrapfig}
\usepackage{amssymb}
\setlength{\topmargin}{-1.5cm}
\setlength{\textwidth}{15.5cm}
\setlength{\textheight}{25.2cm}
\newlength{\minitwocolumn}
\setlength{\minitwocolumn}{0.5\textwidth}
\addtolength{\minitwocolumn}{-\columnsep}
%\addtolength{\baselineskip}{-0.1\baselineskip}
%
\def\Mmaru#1{{\ooalign{\hfil#1\/\hfil\crcr
\raise.167ex\hbox{\mathhexbox 20D}}}}
%
\begin{document}
\newcommand{\fat}[1]{\mbox{\boldmath $#1$}}
\newcommand{\D}{\partial}
\newcommand{\w}{\omega}
\newcommand{\ga}{\alpha}
\newcommand{\gb}{\beta}
\newcommand{\gx}{\xi}
\newcommand{\gz}{\zeta}
\newcommand{\vhat}[1]{\hat{\fat{#1}}}
\newcommand{\spc}{\vspace{0.7\baselineskip}}
\newcommand{\halfspc}{\vspace{0.3\baselineskip}}
\bibliographystyle{unsrt}
\pagestyle{empty}
\newcommand{\twofig}[2]
 {
   \begin{figure}[h]
     \begin{minipage}[t]{\minitwocolumn}
         \begin{center}   #1
         \end{center}
     \end{minipage}
         \hspace{\columnsep}
     \begin{minipage}[t]{\minitwocolumn}
         \begin{center} #2
         \end{center}
     \end{minipage}
   \end{figure}
 }
%%%%%%%%%%%%%%%%%%%%%%%%%%%%%%%%%
%\vspace*{\baselineskip}
\begin{center}
{\Large \bf 平成30年度 共同研究報告書}
\end{center}
\vspace{2mm}
\begin{center}
{\LARGE \bf 
メソスケールシミュレーションによる緩衝材の特性評価に関する研究} 
\end{center}
\begin{center}
岡山大学環境生命科学研究科\\
木本和志
\end{center}
\vspace{10mm}
%%%%%%%%%%%%%%%%%%%%%%%%%%%%%%%%%%%%%%%%%%%%%%%%%%%%%%%%%%%%%%%%
\section{はじめに}
高レベル放射性廃棄物(HLW)の地層処分で用いられるベントナイト緩衝材の,処分環境における
長期的な物理・化学的挙動を予測するには,緩衝材を構成する膨潤性粘土鉱物と間隙水分の
相互作用について理解し,粘土含水系が所定の温度,圧力下でどのような組織構造をとるかを
知る必要がある.粘土鉱物は非常に微細かつ扁平なナノスケールの結晶であり,粘土鉱物間の
間隙も同程度に小さく複雑な形状をしていると考えられている.しかしながら,
鉱物の変形と配置,間隙構造を,顕微鏡下で直接観察することは難しく,
粘土含水系のとるナノからマイクロメートルオーダーの組織構造を明らかにする
ためには,理論的,数値解析的な研究が欠かせない.粘土鉱物と水和水分の相互作用や,
積層した粘土鉱物間隙部における物質拡散や吸着については,量子化学計算や分子動力学計算の方法を用いた
原子,分子スケールでの研究が活発に研究が行われてきた.
一方,緩衝材スケールでの変形や物質輸送は,連続体ベースのモデル化に基づき有限要素法等の
数値解析手法を用いてモデル化やシミュレーションが行われることが普通であり,
分子レベルでモデルとのスケール差は非常に大きい.そのため,原子,分子スケールで得られた
粘土含水系に関する知見を,時空間的なスケールギャップを考慮せず,そのまま連続体スケールに
当てはめることは危険であり,両者の中間スケール(メソスケール)におけるシミュレーションモデルを構築し,
ナノスケールの微視的な物性が,メートルスケールのマクロ挙動にどのように発現するかを
調べる方法が必要とされる.\\

このことを踏まえて本研究では,ナノからマイクロメートルオーダーでの粘土含水系の挙動を
調べるための粗視化メソスケール分子動力学(メソMD)モデルとシミュレーションプログラムの開発を
行う.本研究で開発するメソMD法は,水和粘土分子の単位構造を一つの計算粒子に粗視化して
多体粒子の運動解く分子動力学(MD)シミュレーションの方法で,解くべき多体粒子系の自由度を
削減することで,通常のMDよりも大きな時間,空間的スケールの計算を行うことを目指したものである.
これにより,多数の水和粘土鉱物が凝集,圧縮される過程や,凝集系が外力に
よって変形する挙動を調べることができる.
これまでの研究で開発されたメソMD法プログラムは,水和粘土の単位構造(粗視化粒子)の持つ水分量は
一定であるとの仮定の元でモデル化と計算が行われていた.これは,粘土含水系に水分が均一に
分布し,系が平衡状態にある場合にのみ有効な仮定である.従って,系外から水分が供給される場合や,
初期水分分布に偏りがある場合,非平衡状態にある系の水分分布や応力状態の緩和挙動を調べる場合に
はそのような仮定は成り立たない.\\

本研究は,粘土含水系における水分移動を考慮することのできる,粗視化MDモデルの開発を
行ったもので, 具体的には,粗視化粒子のもつ水和水分の量が時空間的に変動する
することを許容したメソMD技術を提案するものである.
このようなシミュレーション技術は,粘土分子間に明確な気相領域が含まれる不飽和状態での
粘土含水系の挙動を調べるために必要とされる.HLWの地層処分との関係で言えば,ベントナイト
緩衝材は不飽和状態で施工される.そのため,緩衝材が製造,施工直後にどのような水分や密度分布を
持つかという問に答えるには,不飽和状態の粘土の凝集や変形解析を行うことができなければならない.
さらに,廃棄体と緩衝材が処分ピット内に定置された後には,不飽和状態の緩衝材に周辺岩盤から
地下水が浸透する.従って,緩衝材が施工後どのような挙動を示すかを知るには,
膨潤性粘土の不飽和浸透問題を考える必要があり,その結果は緩衝材の長期的挙動に無視できない
影響を与える可能性がある.なお,膨潤性粘土の不飽和水分浸透問題では,水分の浸透に伴い
膨潤応力が発生する.膨潤応力や膨潤ひずみの発生は,粘土含水系の組織構造を変化させるため,
水分の浸透挙動にも影響することから.水分浸透と化学-力学現象である膨潤挙動が連成
した問題になる.そのような複雑な連成現象を,マクロスケールでの挙動観察に基づき
現象論的にモデル化するアプローチ単独では,特に緩衝材の長期挙動予測には十分ではない.
以上のことから,メソMDモデルによるミクロ-メソースケールにおける粘土含水系の挙動解析は,
HLWの地層処分における緩衝材設計と性能評価法を高度化する上で有用な知見を与えることが
できると期待される.\\

以下では,メソMD技術の開発に関する2018年度の取り組みについて報告を行う.はじめに,
粗視化MD法の基礎式を示し,水分移動を伴わない従来のメソMD法について述べる.
次に,水分移動を行うために必要な検討事項について述べ,それらの事項に対する本研究
のアプローチを示す.その後,水分移動を許容する提案モデルの挙動を調べるために行った
数値シミュレーションの結果を示し,水部移動が粘土含水系の圧縮凝集と組織構造形成に
与える影響についてと議論を行う,最後に本研究のまとめと今後の課題を示す. 
%%%%%%%%%%%%%%%%%%%%%%%%%%%%%%%%%%%%%%%%%%%%%%%%%%%%%%
\section{メソスケール粗視化分子動力学(MD)法}
本研究で用いる粗視化メソMD法では,粘土分子の単位構造とそこに水和された水分子を
一つの粗視化粒子として扱う.一つの粘土分子は,分子サイズに応じた数の粗視化粒子を
連結して用いることで表現される.粘土含水系の挙動は,粗視化粒子の集合としての
粘土分子モデルが複数存在して力学的に相互作用をする様を解析することで調べられる.
それぞれの粗視化粒子の位置と速度は,分子内および分子間の相互作用力に従って運動する
ため,粗視化粒子に作用する分子内力,分子間力をモデル化し,運動方程式を数値的積分
することが,メソMD解析に行うべき作業となる.以下,メソMD法で解くべき運動方程式と,
本研究で用いる粗視化粒子間の相互作用モデルを示す.\\

メソMDで扱う粘土分子の数を$M$, 第$m$番目の粘土分子を構成する粗視化粒子の数を$n(m)$と表す.
粗視化粒子数の合計を
\begin{equation}
	N=\sum_{m=1}^M n(m)
	\label{eqn:N_tot}
\end{equation}
とし,各粗視化粒子を識別するために,粒子には1から$N$の通し番号が振られているとする.
時刻$t$における粒子$i$の位置を$\fat{x}_i(t)$,速度を$\fat{v}_i(t)$,質量を$m_i$
とすれば,粒子$i$に関する運動方程式は
\begin{equation}
	m_i \dot {\fat{v}}_i =\fat{f}_i ,\ \ 
	\dot{\fat{x}}_i = \fat{v}_i, \ \ ( i =1,2,\cdots N )
	\label{eqn:eq_mot}
\end{equation}
と表される.ここに,$\fat{f}_i$は,粒子$i$に作用する力を表し,その内訳は
分子間力$\fat{f}_i^{U}$と分子内力$\fat{f}_i^{K}$に分割でき
\begin{equation}
	\fat{f}_i=\fat{f}_i^U+\fat{f}_i^K
	\label{eqn:f_tot}
\end{equation}
と表わされる.第$m$番目の粘土分子を構成する粒子を,粒子番号の集合により
\begin{eqnarray}
	\fat{I}(m) &=&
	\left\{ 
		i_1(m), \, i_2(m), \, \cdots, i_{n(m)}(m)
	\right\}
	 \nonumber \\
	&=&
	\left\{ 
		\left. i_k(m)\right| k=1,\cdots n(m)
	\right\}
	\label{eqn:set_Im}
\end{eqnarray}
と表す.ここで,粒子$i$が属する分子の番号を$m={\cal M}(i)$,分子${\cal M}(i)$を
構成する粒子の中で,粒子$i$が第$k$番目のものであるとき,インデックス$k$を
$k={\cal K}(i)$と書くことにする.一方,分子$m$の第$k$粒子に与えられた,
全粗視化粒子中での通し番号を$i=i(m,k)$と表す.このような粒子の参照方法を用いれば,
メソMD計算に用いる分子内力を次のように書くことができる.
\begin{equation}
	\fat{f}^K_i(\fat{x}_i)=
	-\sum_{j \in \fat{I}(m(i))} K_{ij}
	 \left( \left| \fat{x}_i -\fat{x}_j \right|-r^0_{ij}\right)
	\hat{\fat{r}}_{ij}
	\label{eqn:}
\end{equation}
ここで,$\hat{\fat{r}}_{ij}$は粒子$j$から粒子$i$の方向を指す単位ベクトル
\begin{equation}
	\hat{\fat{r}}_{ij} = \frac{\fat{x}_i-\fat{x}_j}
	{r_{ij}}, \ \ \left( r_{ij}=\left| \fat{x}_i-\fat{x}_j\right|\right)
	\label{eqn:}
\end{equation}
を,$K_{ij}$は第一および第2近接粒子間を連結するバネのバネ定数$K_1$,$K_2$により
\begin{equation}
	K_{ij}=\left\{
	\begin{array}{cc}
		K_1 & \left(\left|{\cal K}(i)-{\cal K}(j)\right|=1\right) \\
		K_2 & \left(\left|{\cal K}(i)-{\cal K}(j)\right|=2\right) \\
		0   & (otherwise)
	\end{array}
	\right.
	\label{eqn:}
\end{equation}
で与えられる定数を意味する.また,$r^0_{ij}$は,分子内力を与える全てのバネが
中立状態にあるとき,すなわち,分子内にひずみが発生していない状態での粒子$i$と
粒子$j$間の距離を表す.
一方,分子間力のモデルには次のようなポテンシャル関数を用いる.
\begin{equation}
	\fat{f}_i^U(\fat{x}_i)
	=
	-\nabla_{x_i} 
	\left\{ 
		\sum_{j=1, \, j \neq i}^N U\left(\fat{x}_i,\,\fat{x}_j; \sigma \right)
	\right\}
	\label{eqn:}
\end{equation}
ここで$U(\fat{x}_i,\fat{x}_j;\sigma)$は,特性距離$\sigma$をパラメータにもつ
レナード-ジョーンズ(LJ)型のポテンシャル関数
\begin{equation}
	U(\fat{x}_i,\fat{x}_j; \sigma) 
	= 4 \varepsilon 
	\left\{ 
	\left(\frac{\sigma}{r_{ij}}\right)^{12}
	-
	\left(\frac{\sigma}{r_{ij}}\right)^6
	\right\}
	\label{eqn:}
\end{equation}
を意味する.\\

以上の方程式に含まれる定数やパラメータの,物理的意味と数値について述べる.
粗視化粒子の設定に従って決まる量には,質量$m_i$と分子間力を与えるバネの
自然長$r^0_{i,i+1}, r^0_{i,i+2}$がある.本研究では,粗視化の単位を粘土分子
の単位構造としているため,モンモリロナイトの分子構造から
\begin{equation}
	m=2.468\times 10 ^{-24}[{\rm kg}]
	, \ \ 
	r^0_{i,i+1}=1[{\rm nm}], \ \ 
	r^0_{i,i+2}=2[{\rm nm}]
	\label{eqn:}
\end{equation}
となる.膨潤状態に応じた層間距離を定めるパラメータ$\sigma$は,モンモリロナイト
についてX線回折試験等の結果から知られている層間距離を参考に,表\ref{tbl:tbl_sig}
に示す範囲で自由に設定することができる.
\begin{table}[h]
	\begin{center}
	\caption{分子間相互作用ポテンシャルにおける特性距離と膨潤状態の対応.}
	\vspace{3mm}
	\begin{tabular}{c||c|c|c|c}
		膨潤状態 & 0層 & 1層 & 2層 & 3層 \\
		\hline
		特性距離$\sigma$[{\rm nm}]& 0.9 & 1.2 & 1.5 & 1.8 \\
	\end{tabular}
	\label{tbl:tbl_sig}
	\end{center}
\end{table}
一方,分子内力を定めるパラメータ$K_1, K_2$と,分子間相互作用の強さを与える
パラメータ$\varepsilon$は,実験や理論的に決定することが難しいため,
別途行った水和粘土の分子動力学計算結果を参考にして,以下の数値を与える.
\begin{equation}
	K_1=2,000[{\rm N/m}], \ \ 
	K_2=4,000[{\rm N/m}], \ \ 
	\varepsilon=1.0\times 10^{-19}[{\rm Nm}]
	\label{eqn:}
\end{equation}
以上のように,本研究の粗視化MD法では,マクロ現象から推定する必要のある
パラメータが含まれていないことが重要な点である.
%
\section{不均一な水和状態を表現するためのモデル}
粘土分子表面に水和した水分の移動を許容する場合,空間的に水分量の偏りが生じる.
そのような状況をメソMD計算上で表現するためには,粗視化粒子毎に水和水の量が
異なることを表現できるモデルが必要になる.前節で述べたように,粗視化粒子に
属する水和水の量は,LJポテンシャルの特性距離$\sigma$の値で制御される.
これまでの粗視化メソMD計算では,$\sigma$の値は全ての粗視化粒子で同一かつ
時間的にも変化しないとしていた.これに対して本研究では,$\sigma$が粒子毎
に異なり,さらに,粘土分子の一方の面と他方の面(表裏面)における水和水分量
が異なる状態も表現できるように,特性距離$\sigma$に以下のように方向依存性
を与える.\\

粒子$i$に属する水分量を表現するための特性距離を$\sigma_i(\theta_i)$と表す.
ここに$\theta_i$は,粒子位置$\fat{x}_i$において,粘土分子表面への
接線方向から測ったポテンシャルの計算点$\fat{y}$の方位を表す.
図\ref{fig:fig9}に示すように,粒子$i$の位置$\fat{x}_i$における
単位接線ベクトルを$\hat{\fat{t}}_i$, 単位法線ベクトルを$\hat{\fat{n}}_i$
とするとき,$\fat{x}_i$から$\fat{y}$を指す単位ベクトルは
\begin{equation}
	\hat{\fat{r}}=\frac{\fat{y}-\fat{x}_i}{\left| \fat{y}-\fat{x}_i\right|}
	\label{eqn:def_rhat}
\end{equation}
と表され,方位$\theta_i$は次の式で与えられる.
\begin{equation}
	\theta_i = {\rm sgn} (\hat{\fat{n}}_i \cdot \fat{r}) 
	\cos^{-1}\left( \hat{\fat{t}}_i\cdot \hat{\fat{r}}\right)
	\label{eqn:def_th}
\end{equation}
なお,粘土分子の表裏面は,単位法線ベクトル$\pm \hat{\fat{n}}_i$の符号で
区別することができるので,以下では$\hat{\fat{n}}$側の水分量を表す特性距離を
$\sigma^+_i$, $-\hat{\fat{n}}$側の水分量を表す特性距離を
$-\sigma^+_i$,と下記,表裏面毎に量の水分を保持した状態を表現できるようにする.
一方, 2つの粗視化粒子間に働く分子間相互作用力を計算するための${\sigma}$の値は,
粘土分子内で隣接する粗視化粒子間の分子間相互作用は無視できると考えられるため,
$\hat{\fat{t}}_i$方向($\theta_i=0,\pi$)の方向では,
$\sigma$を$\sigma_i(0)=\sigma^0=1.35$[nm]となるようにする.
これ以外の方位$\theta_i$では,長軸と短軸径が$\sigma^\pm$と
$\sigma_0$で与えられる楕円で補間して与える.
すなわち,
\begin{equation}
	\left( \frac{\sigma_t}{\sigma^0} \right)^2
	+
	\left( \frac{\sigma_n}{\sigma^{{\rm sgn}(\theta_i)}_i} \right)^2
	=
	1
	\label{eqn:sigma_ellip}
\end{equation}
と
\begin{equation}
	\sigma_t=\sigma(\theta_i)\cos\theta_i, \ \ 
	\sigma_n=\sigma(\theta_i)\sin\theta_i
	\label{eqn:local_basis}
\end{equation}
%\begin{equation}
%	\sigma=\sqrt{\sigma_t^2 +\sigma_n^2}
%	\label{eqn:}
%\end{equation}
とから,
\begin{equation}
	\frac{1}{\sigma_i(\theta_i)}
	=
	\sqrt{
		\left( 
		\frac{\cos\theta_i}{\sigma_0} 
		\right)^2
		+
		\left( 
			\frac{\sin\theta_i}{\sigma_i^{{\rm sgn}(\theta_i)}}
		\right)^2
	}
	\label{eqn:sigma_th}
\end{equation}
によって与える.
粒子$i$と粒子$j$の間で作用する分子間力を計算するときには,
以上の方向依存性を持つ特性距離$\sigma_i,\sigma_j$を,
それぞれの粒子について
\begin{equation}
	\theta_{ij}={\rm sgn} \left(\hat{\fat{n}_i}\cdot\hat{\fat{r}}_{ij} \right) 
	\cos^{-1}(\hat{\fat{t}_i}\cdot \hat{\fat{r}}_{ij})
	\label{eqn:def_thij}
\end{equation}
として,
\begin{equation}
	\sigma_i=\sigma_i(\theta_{ij}), \ \
	\sigma_j=\sigma_j(\theta_{ji})
	\label{eqn:def_thi_thj}
\end{equation}
で計算し,これらの平均:
\begin{equation}
	\bar \sigma= \frac{\sigma_i(\theta_{ij})+\sigma_j(\theta_{ji})}{2}
	\label{eqn:def_thb}
\end{equation}
をLJポテンシャルの特性距離として用いる.このようにすることで,相互作用を計算する
粗視化粒子の相対距離と方向に応じて引力と斥力圏が定まり,
水和水分量に応じた相互作用力を与えることができる.
%--------------------
\begin{figure}[h]
	\begin{center}
	\includegraphics[width=0.5\linewidth]{Figs/fig7.eps} 
	\end{center}
	\caption{
		粗視化粒子の向きを表すために用いる単位接線
		$\hat{\fat{t}}_i,\hat{\fat{t}}_j$
		および, 単位法線ベクトル.
		$\hat{\fat{n}}_i,\hat{\fat{n}}_j$. 
		$\theta_{ij}$は粒子位置$\fat{x}_i$から$\fat{x}_j$を望む方向を,
		$\theta_{ji}$は$\fat{x}_j$から$\fat{x}_i$を望む方向を,
		それぞれの粒子位置における接線ベクトルから測ったときの角度を表す.
	} 
	\label{fig:fig9}
\end{figure}
%--------------------
%\begin{equation}
%	\frac{1}{\sigma_i(\theta_{ij})}
%	=
%	\sqrt{
%		\left( 
%		\frac{\cos\theta_{ij}}{\sigma_0} 
%		\right)^2
%		+
%		\left( \frac{\sin\theta_{ij}}{\sigma_{{\rm sgn}(\theta_{ij})}} \right)^2
%	}
%	\label{eqn:}
%\end{equation}
\section{水和水移動のモデル}
粘土含水系が非平衡状態にあるとき,時間経過に伴い系全体のもつポテンシャルエネルギーが
低下する方向に状態が推移する.従って,系全体のポテンシャルエネルギーが現在の状態よりも
下がる水分の配置をメソMD計算の途上で見つけることができれば,水分移動を考慮しながら,
水和粘土の運動を追跡することができる.メソMD計算における粗視化粒子系全体のエネルギー$E_{tot}$は,各粒子の持つエネルギー$E_i$の総和であるため,
\begin{equation}
	E_{tot}(\fat{\sigma},\fat{V},\fat{X})=\sum_{i=1}^{N}E(i), 
	 \ \ 
	 \left( \fat{\sigma}=\left\{ \sigma^{\pm}_i \right\}
	 ,
	\ \
	 \fat{V}=\left\{ \fat{v}_i\right\}
, \fat{X}=\left\{\fat{x}_i \right\} 
	 \right)
	\label{eqn:Etot}
\end{equation}
と表すことができる.ここで,$\fat{\sigma}$は,全ての粗視粒子の
もつ水分量(系全体での水分分布)を表す$2N$次元のベクトルを,
$\fat{X}$と$\fat{V}$は,粒子系の位置と速度を表す$2N$次元のベクトルである.
式(\ref{eqn:Etot})は,これらを引数に書くことで,全エネルギーが
位置$\fat{X}$,速度$\fat{V}$,水分分布$\fat{\sigma}$に依存する
ことを明示することを意図している.各粗視化粒子のもつエネルギーは,
粘土分子の表面(+面)と裏面(-面)に水和した水分に関連付けられるものの,
2つに分割することができる.すなわち,
\begin{equation}
	E(i)=E^+(i)+E^-(i)
	\label{eqn:sum_Epm}
\end{equation}
と表すことができる.さらに,$E^\pm(i)$は,次のような4つの形態のエネルギーの和
として表すことができる.
\begin{equation}
	E^\pm(i)=E^\pm_{U}(i)+E^\pm_K(i)+E^\pm_{H_2O}(i)+E^\pm_{Surf}(i)
	\label{eqn:Emodes}
\end{equation}
ここに,
$E^{\pm}_U$は,粘土分子間の相互作用力によるポテンシャルエネルギーを,
$E^\pm_K$は運動エネルギーを表す.また,$E^\pm_{H_2O}$は
粘土分子とそれに水和した水分子の相互作用によって生じるポテンシャル
エネルギーを表し,$E^\pm_{Surf}$は水和水の表面自由エネルギーを意味する.
このうち$E^\pm_U(i)$は,異方的LJポテンシャルと式(\ref{eqn:def_thij})を用いて
\begin{equation}
	E_U^\pm(i)=\sum_{j} U(\fat{x}_i,\fat{x}_j)H({\rm sgn}(\pm \theta_{ij}))
	\label{eqn:def_EU}
\end{equation}
によって計算できる.ただし$H(x)$はヘビサイドの単位ステップ関数:
\begin{equation}
	H(x)=\left\{
	\begin{array}{cc}
		1& (x>0)  \\
		0& (x\leq 0) 
	\end{array}
	\right.
	\label{eqn:Step_func}
\end{equation}
である.運動エネルギーは,粒子速度$\fat{v}_i$を用い,
\begin{equation}
	E_K^{\pm}(i)=\frac{1}{2}m^{\pm}(i) |\fat{v}_i|^2 
	\label{eqn:def_EK}
\end{equation}
と書くことができる.ここに,$m^\pm(i)$は,粒子質量$m_i$の分割
\begin{equation}
	m_i=m^+(i)+m^-(i)
	\label{eqn:mi_split}
\end{equation}
を表し,$m^{\pm}$はそれぞれ,$\pm\hat{\fat{n}}_i$側の分子表面に
関連付けることのできる質量を意味する.$m^\pm$は,
無水粘土分子の単位構造が持つ質量を$m_{clay}$, 
水分子1個の質量$m_{H_2O}$,
水和水の分子層数$n^{\pm}$を用いて
\begin{equation}
	m^{\pm}(i)=\frac{m_{clay}}{2}+\frac{3}{2}n_{H_2O}^{\pm}(i)m_{H_2O}
	\label{eqn:def_mpm}
\end{equation}
と書くことができる.なお,$\pm$は着目する粘土分子の面の向きを表し,
係数$3$は,粘土分子単位構造に一層の水分子が水和したときの
単位構造あたりの水分子数である.
水和水と粘土分子の相互作用によるポテンシャルエネルギーに関しては,
水和数$n^\pm_{H_2O}$の関数になると仮定し,
\begin{equation}
	E_{H_2O}^\pm(i)=U_{hyd}
	\left(
		n^\pm_{H_2O}(i)
	\right)
	\label{eqn:def_EH2O}
\end{equation}
と表しておく.$U_{hyd}(n)$は水和数によるエネルギー変化を定める
関数である.粘土分子への水和挙動を$n^\pm$が離散的な値だけをとるとして
モデル化する場合,$U_{hyd}(n)$はデルタ間数列で,
水和数が連続的に変化する場合は,水和数$n$が整数のときに
極小となるような連続関数として与えればよい.本研究では後者のモデルを用いる
ことを検討しているが,後述する数値計算例ではこのエネルギーの変動は考慮していない.
最後に,水分子の表面エネルギーは,気相-水分子相界面の界面自由エネルギー
$\gamma$と,粒子$i$の液相-気相界面$S^\pm_{a/w}(i)$を用いて,
\begin{equation}
	E_{Surf}^{\pm}(i)=\int_{S^\pm_{a/w}(i)}\gamma dS 
	\label{eqn:def_Esurf}
\end{equation}
で与えられる.なお,液相と固相(粘土分子)界面のエネルギーは
$E^\pm_{H_2O}$に含まれると考えておく.
以上のようにして計算される全エネルギーについて,水分分布$\fat{\sigma}$に
関する摂動:
\begin{equation}
	\delta \fat{\sigma}=\left\{ \delta \sigma^\pm_i \right\}
	\label{eqn:var_sig}
\end{equation}
を,全水分量一定:
\begin{equation}
	\sum_{i=1}^N \left( 
		\delta \sigma^+_i
		+
		\delta \sigma^-_i
	\right)
	=0
\end{equation}
の条件において与え,そのときの全エネルギ-$E_{tot}$の変分
\begin{equation}
	\Delta E_{tot}(\fat{\sigma},\fat{V},\fat{X})=
	E_{tot}(\fat{\sigma}+\delta \fat{\sigma},\fat{V},\fat{X})
	-
	E_{tot}(\fat{\sigma},\fat{V},\fat{X})
	\label{eqn:}
\end{equation}
を数値的に計算する.
$\Delta E_{tot}$が負の場合には水分分布を
\begin{equation}
	\fat{\sigma}\rightarrow \fat{\sigma}+\delta \fat{\sigma}
	\label{eqn:sig_update}
\end{equation}
と更新する.運動方程式を数値的に積分する際の適当な時間ステップにおいて,
このような規則に従い水分分布を変更することで,水和粘土分子の運動と変形
だけでなく,粘土分子と水和水の相対的な運動(水和水移動)のシミュレーション
を行うことができる.
%%%%%%%%%%%%%%%%%%%%%%%%%%%%%%%%%%%%%%%%%%%%%%%%%%%%%
\section{数値解析例}
\subsection{水分移動を伴わない場合}
図\ref{fig:fig1}に示すような,200nm$\times$200nmのユニットセルに40個の
粘土分子をもつ粗視化MDモデルを考え,周期境界条件のもと,
一定のひずみ速度で60\%等方的に圧縮する場合を例として数値シミュレーションの
結果を示す.図\ref{fig:fig1}の左は,その際に用いた粘土分子の初期配置を示したもので,
この図の右側には,粘土分子幅の度数分布がヒストグラムとして示されている.
その他,数値シミュレーションに関する主要な計算条件を以下に示す.
\begin{itemize}
\item
ユニットセルサイズ:200$\times$200[nm$^2$]
\item
粘土分子数:M=40,\, 粗視化粒子数:N=2,000
\item
時間ステップ長:$\Delta t=$0.2[ps], 時間ステップ数:50,000ステップ
\item
特性距離:$\sigma^\pm=$1.5[nm](2層膨潤),$\sigma^0=$1.35[nm]
\end{itemize}
第1の数値解析例では,全ての粘土分子表面に水分子が2層水和分だけ水和された状態から
計算をスタートし,圧縮の過程においても,粘土分子の各点に水和した水分の量は一定
としている.図\ref{fig:fig2}は,この場合の結果をスナップショットとして示したもので,
(a)から(f)の順に時間が経過し,ユニットセルのサイズが小さくなっている.
なお,(a)は圧縮開始から80[ps]が経過したときの結果を示し,(f)は1[ns]経過して圧縮が
終了した時点の粘土分子の様子を示している.この図に示されるように,当初分散していた
粘土分子が,ユニットセルの圧縮にともない次第に積層し,粘土層間の他にも
大きな空洞部分ができる様子が見られる.このシミュレーション結果は水分が移動は無く,
従来の粗視化メソMDによる結果と大きく変わるところは無い.ただし,LJポテンシャルは
縦方向と横方向の特性距離$\sigma^\pm$と$\sigma^0$が異なる異方的なものを用いている.
------------------
\begin{figure}[h]
	\begin{center}
	\includegraphics[width=1.0\linewidth]{Figs/model.eps} 
	\end{center}
	\caption{
		粘土分子の初期状態と分子幅の分布(水分量が2層膨潤状態相当で
		均一かつ一定の場合).
	} 
	\label{fig:fig1}
\end{figure}
%--------------------
%--------------------
\begin{figure}[h]
	\begin{center}
	\includegraphics[width=0.9\linewidth]{Figs/fig1.eps} 
	\end{center}
	\caption{
		粘土含水系モデルを等方的に圧縮したときの挙動.水分移動を伴わない場合($\sigma^\pm=1.5$[nm]).
	} 
	\label{fig:fig2}
\end{figure}
%--------------------
\subsection{水分移動を伴う場合}
次に,図\ref{fig:fig1}に示したモデルを用い,水分移動を考慮して行った圧縮凝集
挙動のシミュレーション結果を示す.この計算例では,ランダムに選択した2つの
粗視化粒子の間で水分のやり取りが発生するか否かを,運動方程式の計算スッテップ毎に
粒子数の回数だけ行っている.ただし,1計算ステップでの水分の変化は,
分子間相互作用ポテンシャルの特性距離$\sigma_i^\pm$を$\pm 0.03$[nm]だけ変化させることで行っており,
これは水分子層の1/5の厚さに相当する.また,水分の移動は
\begin{equation}
	\left| \fat{x}_i -\fat{x}_j\right| < 3.5 \bar{\sigma}(i,j)
	\label{eqn:rij_max}
\end{equation}
の条件を満たす粒子$i$と$j$の間でのみ起きると仮定している.
図\ref{fig:fig3}は,以上の条件による計算結果を示したもので,(a)から(f)
の順に時間が経過している.なお,各々のスナップショットに対する
圧縮開始からの経過時間は,図\ref{fig:fig2}と同じである.
水分の移動を考慮した場合も,分散状態にあった粘土分子は次第に積層構造を作り,
圧縮の進展につれ徐々に明確かつ大きな間隙が発生する.また,(a)から(c)までの
時間では,水分の移動を考慮しない場合と非常によく似た結果が得られているが,
後半の(d)から(f)の時間帯では,水分移動を考慮した場合の方が積層数が小さく,
積層した分子の集団がより屈曲していることが分かる.(f)に示した結果は,
シミュレーションの最終段階で圧縮を終了させた時点の結果だが,
計算を継続した場合はこの後も系は緩和を続け,粘土や水分配置は変化する.
しかしながら,水分移動を考慮した場合としない場合の結果は,シミュレーションの
終了時点で明らかに異なり,最終的に安定な状態に至った段階でも大きくことなる
組織構造をとる可能性が高い.
%--------------------
\begin{figure}[h]
	\begin{center}
	\includegraphics[width=0.9\linewidth]{Figs/fig2.eps} 
	\end{center}
	\caption{
		粘土含水系モデルを等方的に圧縮したときの挙動.
		水分移動を伴う場合($\sigma^\pm=1.5$[nm]から計算を開始).
	} 
	\label{fig:fig3}
\end{figure}
%--------------------
\subsection{不均一な初期水分分布をもつ系の圧縮}
粘土分子が分散した状態では,それぞれの分子が異なる量の水和水を持つ状況が
起きると想定する必要がある.その極端なケースとしてここでは,初期状態において
半数の粘土分子が無水の状態にあり,残る半数が3層膨潤状態に相当する水分を持つとして
行った,圧縮凝集シミュレーションの結果を示す.図\ref{fig:fig8}はこのときの,
初期粘土分子の配置を示したものである.分子数はこれまで同様$M=$40としているが,
初期状態での特性距離$\sigma^\pm$が分子によって異なるため,初期状態で
粘土分子間に相互作用が生じないよう、分子サイズや初期配置は前述のモデルと
若干異なるものを用いている.この場合の系の時間的変化を図\ref{fig:fig4}に示す.
初期配置や分子サイズがこれまでの例と異なることには注意が必要であるが,
(d)に示した時間帯から,粘土分子は小さなクラスターを作りはじめ,圧縮終了時点(f)では,
これまでに示した計算例よりも屈曲が大きく,より小さな多数の空洞が形成されている.
このことは,粘土含水系を圧縮凝集させたとき,初期水分配置が,最終的に形成
される積層構造の積層数や層外の間隙サイズに影響することを示唆している.
%--------------
\begin{figure}[h]
	\begin{center}
	\includegraphics[width=1.0\linewidth]{Figs/model2.eps}
	\end{center}
	\caption{
		粘土分子の初期状態と分子幅の分布
		(水分量が0層あるいは3層膨潤状態相当の粘土分子が同数ずつ存在する場合).
	}
	\label{fig:fig8}
\end{figure}
%--------------------
%--------------------
\begin{figure}[h]
	\begin{center}
	\includegraphics[width=0.9\linewidth]{Figs/fig3.eps} 
	\end{center}
	\caption{
		粘土含水系モデルを等方的に圧縮したときの挙動.
		水分移動を伴う場合(初期状態において$\sigma^\pm=0.9$と$1.8$[nm]の
		分子が同数ずつの場合).
	} 
	\label{fig:fig4}
\end{figure}
%--------------------
\subsection{水分分布状況の可視化}
最後に,水分の分布状況を可視化した結果をこれまでの3つの計算ケースそれぞれに
ついて図\ref{fig:fig5}から図\ref{fig:fig7}に示す.おのおの図において,
(a)は圧縮開始から$t=$720[ps]経過した時点の,(b)は圧縮終了時($t=$1[ns]経過時)の
様子を示している.水分移動を考慮していない図\ref{fig:fig5}の結果では,
水分が粘土分子表面に均一に固定的に分布しているため,粘土層間の距離が一様でない
部分では細かな空隙が残っていることが分かる.一方,水分移動を考慮した図\ref{fig:fig6}と
図\ref{fig:fig7}の結果では,積層した粘土の層間には水分が浸透して空隙はほとんど残されていない.
ただし,初期の水分分布状態が異なる図\ref{fig:fig6}と図\ref{fig:fig7}の結果を比べると,
図\ref{fig:fig7}の方が,より水分分布の空間的な変動が大きい.
このことは,圧縮の進行速度に対し,粘土分子が積層構造を発達させるために
必要な水分を移動させる時間を図\ref{fig:fig7}のケースでは
十分に確保することができず、その結果より小さな多数の空洞が残たまま
圧縮が進むことを意味している.
%--------------------
\begin{figure}[h]
	\begin{center}
	\includegraphics[width=0.8\linewidth]{Figs/fig4.eps} 
	\end{center}
	\caption{
		粘土分子と水分の分布状況($\sigma^\pm=1.5$[nm],水分移動を伴わない場合).
	} 
	\label{fig:fig5}
\end{figure}
%--------------------
%--------------------
\begin{figure}[h]
	\begin{center}
	\includegraphics[width=0.8\linewidth]{Figs/fig5.eps} 
	\end{center}
	\caption{
		粘土分子と水分の分布状況(初期状態において$\sigma^\pm=1.5$[nm],
		水分の移動を考慮した場合).
	} 
	\label{fig:fig6}
\end{figure}
%--------------------
%--------------------
\begin{figure}[h]
	\begin{center}
	\includegraphics[width=0.8\linewidth]{Figs/fig6.eps} 
	\end{center}
	\caption{
		粘土分子と水分の分布状況(初期状態において$\sigma^\pm=0.9$と1.8[nm]の
		粘土分子が同数存在し,水分移動を考慮した場合).
	} 
	\label{fig:fig7}
\end{figure}
%--------------------
\section{まとめ}
本年度の研究では,粘土含水系のメソMDシミュレーションにおいて水和水の移動を考慮する
ための必要となるシミュレーション手法を開発し,以下の成果を得た.
\begin{itemize}
\item
不均一な水和水分布を表現するために,異方的なLenard-Jones型分子間力ポテンシャルを開発した.
これにより粘土分子の表裏面を区別するとともに,位置によって異なる水和水の分布を考慮することが
可能となった.
\item
粗視化メソMDにおいて水分移動を考慮する方法として
水分分布の変化に伴うポテンシャルエネルギーの増減を基準として利用する方法を提案した.
この方法では,粘土分子の集合状態や層間の膨潤状態(水分量の多寡),液相表面の形状
の影響を考慮して水分の移動を行うことができる.本年度は,これら影響因子のうち,
特に,粘土分子の相互作用に起因した水分移動を考慮できるシミュレーションを実現した.
\item
以上の方法をメソMDプログラムにおいて実装し,粘土含水系の圧縮凝集シミュレーションを行った.
数値シミュレーションの結果を用い,水分移動を行う場合,行わない場合の比較や
初期水分状態が均一な場合と不均一な場合の比較を行った.
\item
一連の数値シミュレーションの結果から,水分移動を考慮することで,積層した粘土層間に小さな気泡が
残ることの無いより妥当な結果が得られるようになることがわかった.また,初期水分分布に偏りが
ある場合,比較的屈曲の大きな粘土分子の積層構造が形成され,積層構造外の空隙部はあまり大きく
ならないことが分かった.
\item
数値シミュレーションの結果として得られたこれらの知見は,粘土含水系の現実的な組織構造を
再現するためには,水分の移動を考慮することが必要でことを示唆していると考えられる.
\end{itemize}
本年度の研究では,水分移動に関して,粘土分子間の相互作用エネルギー$E_U$のみを考慮している.
そのため,より現実的な組織構造モデルを得るためには,粘土分子と水分子の相互作用エネルギー
$E_{H_2O}$や,液相の界面自由エネルギー$E_{Surf}$も考慮したシミュレーションを行うことが
今後の課題となる.そのためには,$E_{H_2O}$や$E_{Surf}$を分子動力学計算を用いて定量的に
求める必要がある.また,水分移動の頻度や量を,物理化学的な根拠をもって定量的に与えることも
分子動力学計算を援用して取り組むべき重要な課題の一つと言える.最後に,粘土含水系モデルと
シミュレーション結果の妥当性を検証するには,弾性係数や電気化学インピーダンスをはじめとする,
実験で観測できるマクロ物性値との比較を行うことも重要である.
\end{document}


